\section{Discussion}
\label{sec:discussion}

\todo{fill in with lower bounds and conjectures that explain what 
``best possible'' result would look like, and how adversarial and 
stochastic settings compare}

\paragraph{Related work.}
Our setting is closely related to the problem of \emph{peer prediction} 
\citep{miller2005eliciting}, in which we wish to obtain truthful information 
from a population of raters by exploiting inter-rater agreement. 
While several mechanisms have been proposed for these tasks, 
they typically assume that rater accuracy is observable online
\citep{resnick2007influence}, that raters are 
rational agents maximizing a payoff function \citep{dasgupta2013crowdsourced,
kamble2015truth,shnayder2016strong}, that the workers follow a simple 
statistical model \citep{karger2014budget,zhang2014crowdsourcing,
zhou2015regularized}, or some combination of the above \citep{shah2015double,
shah2015approval}. 

In crowdsourcing, it is common to use ``gold sets'' -- questions where the 
ground truth is known -- to assess worker performance. 
However, there are tasks for which gold sets do not make sense
--- for instance, ``draw an interesting picture'' or ``translate this sentence'', 
where there are different equally good answers. In this case, the only way to 
evaluate quality is via evaluation by other workers or by the manager of 
the experiment. Even for more straightforward tasks, \citet{vuurens2011spam} 
suggest that workers may be able to identify gold set questions in some 
instances. Beyond crowdsourcing, there are settings where human judgment is 
necessary and where there are documented attempts to game the system --- 
\citet{harmon2004amazon} reports that many authors and filmmakers 
engage in dishonest tactics to inflate their reviews on amazon.com, while 
\citet{priedhorsky2007creating} find that roughly $5\%$ of revisions on 
wikipedia are damaged in some way (e.g. by vandalism). In such settings where 
content is primarily moderated by the crowd, building resilient ratings 
mechanisms is important for maintaining quality.

The work most closely related to ours is that of \citet{christiano2014provably,
christiano2016robust}, who studies online collaborative prediction in 
the presence of adversaries; roughly, when raters interact with an item 
they predict its quality and afterwards observe the actual quality; the 
goal is to minimize the number of incorrect 
predictions among the honest raters. This differs from our setting in that 
(i) the raters are trying to learn the item qualities as part of the task, 
and (ii) there is no requirement to induce a final global estimate of the 
high-quality items, which is necessary for estimating quantiles.
It seems possible however that there are theoretical ties between this 
setting and ours, which would be interesting to explore.
