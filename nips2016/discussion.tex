\section{Open Directions and Related Work}
\label{sec:discussion}
\iffalse
\todo{fill in with lower bounds and conjectures that explain what 
``best possible'' result would look like, and how adversarial and 
stochastic settings compare}\fi

\paragraph{Future Directions}
On the theoretical side, perhaps the most immediate open question is whether it is 
possible to improve the dependence of $k$ (the amount of work required per worker) 
on the parameters $\alpha, \beta,$ and $\eps$.  It is tempting to hope that 
a tight result would have 
$k = \Theta\p{\frac{\log(1/\alpha)}{\min (\alpha,\beta)\eps^2}}$, in loose analogy
to recent results for the stochastic block model \citep{banks2016information}.


More generally, there are a number of other parameter regimes and settings that seem worth considering.  While we focus on the regime where the number of items exceeds the number of workers, it is also worth considering the regime where the number of workers exceeds the number of items:  in such a regime, is it sufficient to have each worker review only $2/\alpha$ items?  If items can be assigned to workers in an adaptive setting (dependent on the ratings we have already seen), how much can we reduce the amount of work?  What is the tradeoff between the total amount of work required across all workers, and the non-uniformity in the amount of work assigned to each worker?  (For example, in the extreme setting where we  are willing to assign all $m$ items to $O(1/\alpha)$ workers, then the total required work is only $O(m/\alpha)$.)



\paragraph{Related work.}
Our setting is closely related to the problem of \emph{peer prediction} 
\citep{miller2005eliciting}, in which we wish to obtain truthful information 
from a population of raters by exploiting inter-rater agreement. 
While several mechanisms have been proposed for these tasks, 
they typically assume that rater accuracy is observable online
\citep{resnick2007influence}, that raters are 
rational agents maximizing a payoff function \citep{dasgupta2013crowdsourced,
kamble2015truth,shnayder2016strong}, that the workers follow a simple 
statistical model \citep{karger2014budget,zhang2014crowdsourcing,
zhou2015regularized}, or some combination of the above \citep{shah2015double,
shah2015approval}. 

%In crowdsourcing, it is common to use ``gold sets'' -- questions where the 
%ground truth is known -- to assess worker performance. 
%However, there are tasks for which gold sets do not make sense
%--- for instance, ``draw an interesting picture'' or ``translate this sentence'', 
%where there are different equally good answers. In this case, the only way to 
%evaluate quality is via evaluation by other workers or by the manager of 
%the experiment. Even for more straightforward tasks, \citet{vuurens2011spam} 
%suggest that workers may be able to identify gold set questions in some 
%instances. Beyond crowdsourcing, there are settings where human judgment is 
%necessary and where there are documented attempts to game the system --- 
%\citet{harmon2004amazon} reports that many authors and filmmakers 
%engage in dishonest tactics to inflate their reviews on amazon.com, while 
%\citet{priedhorsky2007creating} find that roughly $5\%$ of revisions on 
%wikipedia are damaged in some way (e.g. by vandalism). In such settings where 
%content is primarily moderated by the crowd, building resilient ratings 
%mechanisms is important for maintaining quality.

The work most close to ours is \citet{christiano2014provably,
christiano2016robust}, which studies online collaborative prediction in 
the presence of adversaries; roughly, when raters interact with an item 
they predict its quality and afterwards observe the actual quality; the 
goal is to minimize the number of incorrect 
predictions among the honest raters. This differs from our setting in that 
(i) the raters are trying to learn the item qualities as part of the task, 
and (ii) there is no requirement to induce a final global estimate of the 
high-quality items, which is necessary for estimating quantiles.
It seems possible however that there are theoretical ties between this 
setting and ours, which would be interesting to explore.
