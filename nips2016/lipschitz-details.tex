\subsection{Details of Lipschitz Bound (Equation \ref{eq:lipschitz-usage})}
\label{sec:lipschitz-details}

The proof essentially consists of matching up each value $\ravg_j$, for 
$j \in T^*$, with a set of values $\ravg_{j'}$, $j' \geq j$, where the 
corresponding $\M_{i,j'}$ sum to $1$; we can then invoke the condition 
\eqref{eq:lipschitz}. Unfortunately, expressing this idea 
formally is a bit notationally cumbersome.

Before we start, we observe that the Lipschitz condition \eqref{eq:lipschitz} 
implies that, if $\ravg_j \geq \ravg_{j'}$, then 
$\ravg_j - \ravg_{j'} \leq L \cdot \p{\Aavg_{i,j} - \Aavg_{i,j'}} + \epsilon_0$. 
It is this form of \eqref{eq:lipschitz} that we will make use of below.

Now, let $I_j = \bI[j \in T^*]$, and without loss of generality suppose that 
the indices $j$ are such that $\ravg_1 \geq \ravg_2 \geq \cdots \geq \ravg_m$. 
For a vector $v \in [0,1]^m$, define
\begin{equation}
\label{eq:def-h}
h(\tau, v) \eqdef \inf\{j \mid \sum_{j' = 1}^j v_{j'} \geq \tau\},
\end{equation} 
where $h(\tau,v) = \infty$ if no such $j$ exists.
We observe that for any vector $v \in [0,1]^m$, we have 
\begin{equation}
\label{eq:sum-integral}
\sum_{j \in [m]} v_j\ravg_j = \int_{0}^{\infty} \ravg_{h(\tau; v)} d\tau,
\end{equation}
where we define $\ravg_{\infty} = 0$ (note that the integrand is therefore $0$ 
for any $\tau \geq \|v\|_1$). Hence, we have
\begin{align}
\sum_{j \in T^*} \ravg_j - \sum_{j \in [m]} \M_{i,j}\ravg_j 
 &= \sum_{j \in [m]} I_j\ravg_j - \sum_{j \in [m]} \M_{i,j}\ravg_j \\
 &= \int_{0}^{\beta m} \ravg_{h(\tau, I)} - \ravg_{h(\tau, \M_i)} d\tau \\
 &\stackrel{(i)}{\leq} \int_{0}^{\beta m} \left[L \cdot \p{\Aavg_{h(\tau, I)} - \Aavg_{h(\tau, \M_i)}} + \epsilon_0\right] d\tau \\
 &= L\cdot \p{\sum_{j \in [m]} I_j\Aavg_j - \sum_{j \in [m]} \M_{i,j}\Aavg_j} + \beta m\epsilon_0 \\
 &= L\cdot \p{\sum_{j \in T^*} \Aavg_j - \sum_{j \in [m]} \M_{i,j}\Aavg_j} + \beta m\epsilon_0,
\end{align}
which implies \eqref{eq:lipschitz-usage}. The key step is (i), 
which uses the fact that $h(\tau,I) \leq h(\tau, \M_i)$ (because $I$ is 
maximally concentrated on the left-most indices of $[m]$), and hence 
$\ravg_{h(\tau,I)} \geq \ravg_{h(\tau, \M_i)}$.
