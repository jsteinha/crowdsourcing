\begin{figure}
\begin{center}
\begin{tikzpicture}[scale=5/6, every node/.style={scale=5/6}]
\def\s{6.7}
\def\x{0.3*\s};
\def\y{0.2*\s};
\def\X{1*\s};
\def\Y{0.9*\s};
\def\pux{0.57*\s};
\def\puy{\Y};
\def\Msx{\X};
\def\Msy{0.7*\s};
\def\Bx{\Msx+0.3*\s};
\def\By{\Msy+0.12*\s};
\def\plx{0.8*\s};
\def\ply{\y};
\def\ex{\X-0.14*\s}
\coordinate (O) at (\x,\y);
\coordinate (Cx) at (\X,\y);
\coordinate (Cy) at (\x,\Y);
\coordinate (C) at (\X,\Y);
\coordinate (x-ax) at (\X+0.4,\y);
\coordinate (y-ax) at (\x,\Y+0.4);
\coordinate (p-l) at (\plx,\ply);
\coordinate (p-u) at (\pux,\puy);
\coordinate (M-star) at (\Msx,\Msy);
\coordinate (B) at (\Bx,\By);
\def\Bpx{{\Bx-(\By-\Msy)/(\Msx-\pux)*(\puy-\Msy)}};
\def\Bpy{\Msy};
\coordinate (Bp) at (\Bpx,\Bpy);
\draw[opacity=0,name path=l--c] (M-star) -- (p-l);
\draw[opacity=0,name path=l--b] (p-u) -- (M-star);
\draw[opacity=0,name path=l--a] (\ex,\X) -- (\ex,\x);
\path [name intersections={of=l--a and l--b,by=e1}];
\path [name intersections={of=l--a and l--c,by=e2}];
\fill [opacity=0.3,gray] (O) -- (Cy) -- (C) -- (Cx) -- cycle;
\fill [opacity=0.3,blue]
  (O) -- (Cy) -- (p-u) -- (M-star) -- (p-l) -- cycle;
\draw (p-u) edge[thick,blue!70!black] node[pos=0.3,below,sloped] {$\|M\|_* \leq \rho$} (M-star);
\fill [opacity=0.3,red] (e1) -- (M-star) -- (e2) -- cycle;
\draw (M-star) edge[thick,blue!70!black] (p-l);
\draw (O) edge[thick,->] (x-ax);
\draw (O) edge[thick,->] (y-ax);
\node[below=1.2em of Cx] (Cx-s) {};
\node[above=1.2em of C] (C-n) {$M_{\good} = R_{\good}$};
\draw (C-n) edge[dashed] node[pos=0.1,sloped,above] {} (Cx-s);
\draw (Cx) edge (C);
\draw (Cy) edge (C);
\fill(M-star) circle(1.5pt);
\draw (M-star) edge[->] (B);
\draw (M-star) edge[->] (Bp);
\node[right=-0.1em of B] {$B$};
\node[right=-0.1em of Bp] {$B_{\good}\!-\!Z_{\good}$};
\node[right=-0.1em of x-ax] {$M_{\good}$};
\node[above=-0.1em of y-ax] {$M_{\bad}$};
\node[below left=-0.3em and -0.1em of M-star,scale=0.9] {$M^*$};
\draw (\ex,\Y+0.08*\s) edge[dashed,red!70!black] node[pos=0.47,above,rotate=90,scale=0.68,red!50!black] {$\langle B_{\good}\!-\!Z_{\good}, M^*\!-\!M \rangle \leq \epsilon$} (\ex,\y-0.09*\s);
\coordinate[below right=0.65em and 0.45em of e1] (Mt);
\fill(Mt) circle(1.5pt);
\node[above right=0em and 0em of Mt,scale=0.9] {$\M$};
\end{tikzpicture}
\end{center}
\vskip -0.15in
\caption{Illustration of our Lagrangian duality argument, and of the role 
of $Z$. The blue region represents the nuclear norm constraint and the gray 
region the remaining constraints. Because the blue region slopes downwards, 
a decrease in $M_{\good}$ can be offset by an increase in $M_{\bad}$ when 
measuring $\langle B, M \rangle$. The vector $B-Z$ accounts for this offset, 
and the red region represents the constraint 
$\langle B_{\good}-Z_{\good}, M^*_{\good} - M_{\good} \rangle \leq \epsilon$, which is guaranteed to contain $\M$.
}
\vskip 0.05in
\label{fig:lagrangian}
\end{figure}
