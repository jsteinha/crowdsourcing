\begin{figure}
\begin{tikzpicture}[scale=5/6, every node/.style={scale=5/6}]
\def\s{6}
\def\x{0*\s};
\def\y{0*\s};
\def\X{1*\s};
\def\Y{1*\s};
\def\pux{0.5*\s};
\def\puy{\Y};
\def\Msx{\X};
\def\Msy{0.7*\s};
\def\Bx{\Msx+0.3*\s};
\def\By{\Msy+0.12*\s};
\def\plx{0.7*\s};
\def\ply{\x};
\coordinate (O) at (\x,\y);
\coordinate (Cx) at (\X,\y);
\coordinate (Cy) at (\x,\Y);
\coordinate (C) at (\X,\Y);
\coordinate (x-ax) at (\X+0.4,\y);
\coordinate (y-ax) at (\x,\Y+0.4);
\coordinate (p-l) at (\plx,\ply);
\coordinate (p-u) at (\pux,\puy);
\coordinate (M-star) at (\Msx,\Msy);
\coordinate (B) at (\Bx,\By);
\def\Bpx{{\Bx-(\By-\Msy)/(\Msx-\pux)*(\puy-\Msy)}};
\def\Bpy{\Msy};
\coordinate (Bp) at (\Bpx,\Bpy);
\fill [opacity=0.3,gray] (O) -- (Cy) -- (C) -- (Cx) -- cycle;
\fill [opacity=0.3,blue]
  (O) -- (Cy) -- (p-u) -- (M-star) -- (p-l) -- cycle;
\draw (p-u) edge[thick,blue!70!black] node[midway,below,sloped] {$\|M\|_* \leq \rho$} (M-star);
\draw (M-star) edge[thick,blue!70!black] (p-l);
\draw (O) edge[thick,->] (x-ax);
\draw (O) edge[thick,->] (y-ax);
\node[below=1.2em of Cx] (Cx-s) {};
\node[above=1.2em of C] (C-n) {$M_{\good} = R_{\good}$};
\draw (C-n) edge[dashed] node[pos=0.1,sloped,above] {} (Cx-s);
\draw (Cx) edge (C);
\draw (Cy) edge (C);
\fill(M-star) circle(1.5pt);
\draw (M-star) edge[->] (B);
\draw (M-star) edge[->] (Bp);
\node[right=-0.1em of B] {$B$};
\node[right=-0.1em of Bp] {$B_{\good}\!-\!Z_{\good}$};
\node[right=-0.1em of x-ax] {$M_{\good}$};
\node[above=-0.1em of y-ax] {$M_{\bad}$};
\node[below left=-0.3em and -0.1em of M-star] {$M^*$};

\end{tikzpicture}
\caption{Illustration of our Lagrangian duality argument, and of the role 
of $Z$. We first optimize $\langle B, M \rangle$ over $\sP$ while constraining 
$M_{\good} = R_{\good}$, thus obtaining $M^*$. We would like to bound how far 
the optimum would move if we remove the constraint $M_{\sC} = R_{\sC}$; let 
$M'$ be this new optimum. }
\label{fig:lagrangian}
\end{figure}
