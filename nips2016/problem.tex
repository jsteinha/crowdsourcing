\section{Problem statement}
\label{sec:assumptions}

% setting to keep in mind: rating matrix A
% observe d entries in each row at random, scale up by n/d
We will start with a running example to keep in mind, and 
then state the general formal properties needed for our 
results to hold. At the end, we will verify that these properties 
apply to our running example, as well as to some other situations 
of interest such as the stochastic block model.

\paragraph{Running Example.}
We suppose that there is a matrix 
$\Aavg \in [-1,1]^{n \times m}$ of ratings, such that $\Aavg_{ij}$ 
denotes the rating that person $i$ assigns to item $j$. 
Moreover, we assume that there is a ``true'' rating vector $\rtrue \in [-1,1]^m$.
Rather than observing $\Aavg$ and $\rtrue$, we observe noisy, scaled-up 
versions $\Aobs$ and $\robs$:
\[ \Aobs_{ij} = \left\{ \begin{array}{ccl} \frac{n}{k}\Aavg_{ij} & : & \text{ with probability $\frac{k}{n}$} \\ 0 & : & \text{ else} \end{array} \right., \quad\quad 
  \robs_j = \left\{ \begin{array}{ccl} \frac{n}{k'}\rtrue_j & : & \text{ with probability $\frac{k'}{n}$} \\ 0 & : & \text{ else} \end{array} \right.. \] %}}
Note that the scaling ensures that $\bE[\Aobs] = \Aavg$ and $\bE[\robs] = \rtrue$. 

We further assume that there is a set $\sC \subseteq [n]$ of good raters such 
that $\Aavg_{ij} = \rtrue_j$ for all $i \in \sC$, $j \in [m]$. Our goal is to 
recover the $\beta$-quantile of $\rtrue$; in other words, letting $Q_{\beta}$ 
denote the set of $\beta m$ indices $j$ for which $\rtrue$ is largest, we 
wish to recover a set $S$ of size $\beta m$ such that
\[ \frac{1}{\beta m}\p{\sum_{j \in S} \rtrue_j - \sum_{j \in Q_{\beta}} \rtrue_j} \geq - \epsilon. \]

\paragraph{General Setting}
The example above is limited in a few ways; first, it makes a very specific 
assumption about the randomness in $\Aobs$ and $\robs$. Second, it implicitly 
assumes that the adversarial actions are chosen independently of which items 
they are assigned to rate, which is implausible. Finally, it assumes that 
$\bE[\Aobs_i] = \rtrue$ for all $i \in \sC$, which among other things rules out 
having noisy raters with different levels of noise for each rater. We will end 
up relaxing to the assumption that each good rater shares the same 
$\beta$-quantile $Q_{\beta}$; one can likely relax this still further but that 
is beyond the scope of this paper.\footnote{For instance, we believe that 
a sufficient condition is that the union of the $\beta$-quantiles of the good 
users has size at most $\beta m \cdot \p{1 + 1/\sqrt{|\sC|}}$.}

For a matrix $X$, 
we will let $X_{\good}$ denote the rows of $X$ indexed by $\good$, 
and $X_{\bad}$ the remaining rows. We also assume that $\good$ is a constant 
fraction of $[n]$: $|\good| \geq \goodfrac n$ for some $\goodfrac$.

% assumptions:
% set of good users C
% ``expected ratings'' A_{\sC}
%   - all share same \beta-quartile
%   - all assign average score at least 1/2 to it
% ``actual ratings'' \A_{\sC}
%   - \|A_{\sC} - \A_{\sC}\|_op \leq n / sqrt(fac)
%   - \Au_{\bad}
The important properties of $\Aobs$ are summarized below:
\begin{assumption}
\label{ass:A}
For some $\fac > 0$, the matrix $\Aobs$ satisfies the following properties:
\begin{enumerate}
\item There exists a matrix $\Aavg \in [-\infty,1]^{n \times m}$ such that 
      $\|\Aobs - \Aavg\|_{\op} \leq \frac{m}{\sqrt{\fac}}\max\p{1, \sqrt{\frac{\alpha n}{m}}}$.
\item There exists a set $\good \subseteq [n]$ of size at least $\goodfrac n$ 
      such that for each $i \in \good$, the $\beta$-quantile of $\Aavg_i$ is 
      $Q_{\beta}$, and furthermore
      \[ \frac{1}{\beta m} \sum_{j \in Q_{\beta}} \Aavg_{ij} \geq \frac{1}{2}. \]
\end{enumerate}
\end{assumption}
(The constant $\frac{1}{2}$ can be replaced with any positive constant.)
\begin{assumption}
\label{ass:r}
The observed rating vector $\rtrue$ satisfies \todo{include condition}
\end{assumption}
The condition $\|\Aobs - \Aavg\|_{\op} \leq \frac{m}{\sqrt{\fac}}$ is what 
we would typically get out of matrix concentration bounds (see e.g. \citet{todo}), 
where each row of $\Aobs$ consists of subsampling $k = \Theta(\fac)$ entries 
of $\Aavg$ at random. However, because the adversary's strategy can depend 
on which entries we choose, some care is needed. The key observation that limits 
the influence of the adversary is that (at least according to Assumption~\ref{ass:A}), $\Aavg_{\bad}$ need not correspond to any sort of expected value, 
and the only constraint it need satisfy is that 
$\max_{i \in \bad, j \in [m]} \Aavg_{ij} \leq 1$. As a result, the worst 
the adversary can do is make $\Aobs_{ij}$ is as large as possible, i.e. 
$\Aobs_{ij} = \frac{m}{k}\bI[(i,j) \in \Obs]$ for all $i \in \bad$.

This general setting captures the running example for $k = \Theta(\fac)$:
\begin{proposition}
Suppose that $\Aobs_{ij} = \frac{m}{k}E_{ij}\Anom_{ij}$, 
where the $E_{ij}$ i.i.d. $\operatorname{Bernoulli}(\frac{k}{m})$, and 
$\Anom_{ij}$ is the rating that person $i$ assigns to item $j$. Further 
assume that $\bE[\Anom_{i}] = \Aavg_i$, for $\Aavg_i$ satisfying 
Assumption~\ref{ass:A}, and that the $\Anom_{ij}$ are jointly independent 
for $i \in \good$, $j \in [m]$. (The $\Anom_{ij}$ for $i \in \bad$ can be 
arbitrarily dependent on each other, as well as on $E$ and on 
$\Anom_{\good}$.)

%
%Suppose that the following scheme is carried out: 
%for each $i,j \in [n] \times [m]$, with probability $\frac{k}{m}$ person 
%$i$ is asked to supply a rating for item $j$ that lies in $[-1,1]$. 
%Then $\Aobs_{ij}$ is set to be the rating scaled by $\frac{m}{k}$, or else 
%$0$ if we did not ask for a rating. When choosing a response, the 
%users reply with rating whose expectation is $\Aavg_{ij}$, where 
%$\Aavg_i$ satisfies Assumption~\ref{ass:A}. The dishonest users reply 
%with an output that may depend on the ratings of the honest users, the 
%randomness in the procedure, and the ratings of the other dishonest users.

Then, assuming that $k \geq \fac + \sqrt{\fac \log(1/\delta)}$, 
Assumption~\ref{ass:A} holds with probability $1-\delta$, where $\Aavg_i$ is 
as defined above for $i \in \good$, and 
$\Aavg_{ij} = 1 + \frac{m}{k}E_{ij}(\Anom_{ij} - 1)$ for $i \in \bad$.
\end{proposition}
