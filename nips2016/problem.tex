\section{Problem statement}
\label{sec:assumptions}

Before stating our assumptions formally, we give a concrete example that our 
assumptions are meant to capture. 
We suppose that there is an unobserved matrix $\Aavg \in [0,1]^{n \times m}$
of ratings, such that $\Aavg_{ij}$ denotes the rating that person $i$ would 
assign (in expectation) to item $j$. 
We also let $\rtrue \in [0,1]^m$ denote the unobserved vector 
specifying our own ratings. Rather than observing $\Aavg$ and $\rtrue$, 
we observe noisy, scaled-up versions $\Aobs$ and $\robs$:
\begin{equation}
\label{eq:sampling}
\Aobs_{ij} = \left\{ \begin{array}{ccl} \frac{n}{k}\Aavg_{ij} & : & \text{ with probability $\frac{k}{n}$} \\ 0 & : & \text{ else} \end{array} \right., \quad\quad 
  \robs_j = \left\{ \begin{array}{ccl} \frac{n}{k_0}\rtrue_j & : & \text{ with probability $\frac{k_0}{n}$} \\ 0 & : & \text{ else} \end{array} \right..
\end{equation} %}}
Note that the scaling ensures that $\bE[\Aobs] = \Aavg$ and $\bE[\robs] = \rtrue$. 

For a vector $v \in \bR^m$, let $Q_{\beta}(v)$ denote the \emph{$\beta$-quantile} 
of $v$ --- that is, the $\beta m$ coordinates $j$ for which $v_j$ is largest.
We assume that there is a set $\good \subseteq [n]$ of good raters of size 
at least $\alpha n$ such 
that $Q_{\beta}(\Aavg_{i}) = Q_{\beta}(\rtrue)$ for all $i \in \good$, and our 
goal is to recover $\beta$-quantile of $\rtrue$. More precisely, we wish to 
recover a set $T$ of size $\beta m$ such that
\[ \frac{1}{\beta m}\p{\sum_{j \in T} \rtrue_j - \sum_{j \in Q_{\beta}(\rtrue)} \rtrue_j} \geq - \epsilon. \]

\paragraph{Notation.}
For a matrix $X$, we will let $X_{\good}$ denote the rows of 
$X$ indexed by $\good$, and $X_{\bad}$ the remaining rows. 
We will let $\|X\|_{\op}$ denote the operator norm of $X$ and 
$\|X\|_*$ denote the nuclear norm (sum of singular values) of $X$.
%We assume that $\good$ is a constant fraction of $[n]$ and let 
%$\goodfrac$ denote this fraction: i.e., $|\good| \geq \goodfrac n$.

\paragraph{General Setting.}
The example above is limited in a few ways; first, it makes a very specific 
assumption about the randomness in $\Aobs$ and $\robs$. Second, it implicitly 
assumes that the adversaries' actions are chosen independently of which items 
they are assigned to rate, which is implausible. Third, it assumes that 
$\bE[\Aobs_i] = \rtrue$ for all $i \in \good$, which among other things rules out 
having noisy raters with different levels of noise for each rater.
Finally, it assumes that our own ratings $\rtrue$ are perfect, whereas in reality 
we would like to tolerate some level of error on our own part as well.
Below, we will first lay out assumptions on $\Aobs$, and then on $\robs$, 
under which our results will hold.

\paragraph{Assumptions on $\Aobs$.}
We assume that $\Aobs$ is ``close'' to a matrix 
$\Aavg \in [-\infty,1]^{n \times m}$ in the sense that 
$\|\Aobs - \Aavg\|_{\op} \leq \frac{m}{\sqrt{\fac}}\max\p{1, \sqrt{n/m}}$. 
For instance, this holds with high probability if $\Aobs$ is generated as in 
\eqref{eq:sampling}, for $k = \Omega(\fac)$. Some care is needed because the 
adversary is allowed to adapt to the random sampling, but we will see in 
Proposition~\ref{prop:sampling-reduction} below that this can be dealt with.

The important properties of $\Aavg$ are summarized below.
% assumptions:
% set of good users C
% ``expected ratings'' A_{\sC}
%   - all share same \beta-quartile
%   - all assign average score at least 1/2 to it
% ``actual ratings'' \A_{\sC}
%   - \|A_{\sC} - \A_{\sC}\|_op \leq n / sqrt(fac)
%   - \Au_{\bad}
First, we assume that the $\beta$-quantile of $\Aavg$ is correct:
\begin{assumption}[Shared Quantiles]
\label{ass:quantile}
For each $i \in \good$, the row vector $\Aavg_i$ lies in $[0,1]^m$ and satisfies 
$Q_{\beta}(\Aavg_i) = Q_{\beta}(\rtrue)$.
\end{assumption}
%In most settings, standard matrix concentration bounds will imply that 
%$\fac$ is proportional to the number of entries of $\Aobs$ that are observed 
%in each row. 
It seems possible to relax Assumption~\ref{ass:quantile} to only 
require approximate agreement of the $\beta$-quantiles, but that is beyond the 
scope of this paper.\footnote{For instance, we believe that 
a sufficient condition is that the union of the $\beta$-quantiles of the good 
users has size at most $\beta m \cdot \p{1 + 1/\sqrt{|\good|}}$.}

Next, we assume that our rating is a Lipschitz function of the average 
honest rating:
\begin{assumption}[Lipschitz ratings]
\label{ass:lipschitz}
There exists a constant $L$ such that 
\[ \rtrue_j - \rtrue_{j'} \leq \frac{L}{|\good|}\sum_{i \in \good} \Aavg_{ij} - \Aavg_{ij'} \]
for all $j \in Q_{\beta}(r^*), j' \not\in Q_{\beta}(r^*)$.
\end{assumption}
Assumption~\ref{ass:lipschitz} is important because it implies that 
near-optimal solutions with respect to the $\Aavg_i$ are also near-optimal 
with respect to $\rtrue$.
To avoid excessive notation, we will assume that $L = \oo\p{1}$ throughout 
the paper (say, $L \leq 3$). To recover the dependence on $L$, simply replace 
all instances of the error $\error$ in the results below with $\error/L$.

We now return to a generalization of our example from the beginning of the 
section, and show that the bound on $\|\Aobs - \Aavg\|_{\op}$ holds for an 
appropriately chosen $\Aavg$.
\begin{proposition}
\label{prop:sampling-reduction}
Suppose that $\Aobs_{ij} = \frac{m}{k}E_{ij}\Anom_{ij}$, 
where the $E_{ij}$ are i.i.d.~$\operatorname{Bernoulli}(\frac{k}{m})$, and 
$\Anom_{ij}$ is the rating that person $i$ assigns to item $j$. Further 
assume that for each $i \in \good$, $\bE[\Anom_{i}] = \Aavg_i$, 
for $\Aavg_i$ satisfying Assumptions~\ref{ass:quantile} and \ref{ass:lipschitz},
 and that the $\Anom_{ij}$ are jointly independent for 
$i \in \good$, $j \in [m]$. (The $\Anom_{ij}$ for $i \in \bad$ can be 
arbitrarily dependent on each other, as well as on $E$ and on 
$\Anom_{\good}$.)

Then, assuming that $k \geq \fac + \sqrt{\fac \log(1/\delta)}$, 
$\|\Aobs - \Aavg\|_{\op} \leq \frac{n}{\sqrt{\fac}}\max\p{1, \sqrt{n/m}}$
with probability $1-\delta$, where $\Aavg_i$ is 
as defined above for $i \in \good$, and 
$\Aavg_{ij} = 1 - \frac{m}{k}E_{ij}(1 - \Anom_{ij})$ for $i \in \bad$.
\end{proposition}
Note how $\Aavg_i$ is defined for $i \in \bad$; since there are no constraints 
on $\Aavg_{\bad}$ except that its entries be bounded above by $1$, we have a 
great deal of freedom to match $\Aobs_{\bad}$. In particular, $\Aavg_{\bad}$ 
as defined above cancels out the adversaries' action, setting 
$(\Aobs - \Aavg)_{ij}$ to $\frac{m}{k}E_{ij} - 1$, which is entirely independent 
of the adversaries and satisfies known matrix concentration bounds.
This is one of two key constructions in our analysis that allows us to bound 
the influence of the adversaries. The other is Proposition~\ref{prop:subgradient}, 
which will be explained in more detail in Section~\ref{sec:approach}.

\paragraph{Assumptions on $\robs$.}
The conditions that $\robs$ needs to satisfy are substantialy simpler, and 
are given in the condition below:
\begin{assumption}
\label{ass:r}
For some $\fac_0$, the observed rating vector $\robs$ satisfies 
$\|\robs - \rtrue\|_{\infty} \leq \frac{m}{\fac_0}$ almost surely, 
$\bE[\|\robs-\rtrue\|_2^2] \leq \frac{m^2}{\fac_0}$, 
and the coordinates of $\robs$ are jointly independent.
\end{assumption}
If we suppose that 
$\robs_j = \frac{m}{k_0}I_j\rtrue_j$, where $I_j$ is $\operatorname{Bernoulli}\p{\frac{k_0}{m}}$, then Assumption~\ref{ass:r} holds for $k = \fac_0$.

\paragraph{Main result.}
With Assumptions~\ref{ass:quantile}, \ref{ass:lipschitz}, and \ref{ass:r} 
in place, we can now state our main results. Our algorithm and also our results 
can be broken up into two parts: first, we approximately recover the 
$\beta$-quantiles for each of the raters, in the form of a matrix 
$\M \in [0,1]^{n \times m}$ such that $\M_i$ is close to the uniform distribution 
over $Q_{\beta}(\Aavg_i)$ for each $i \in \good$. Second, given such a matrix, 
we recover a set $T$ that is close to the $\beta$-quantile of $\rtrue$. 
We state these results formally below:
\begin{theorem}
\label{thm:main-M}
Suppose that 
$\|\Aobs - \Aavg\|_{\op} \leq \frac{m}{\sqrt{\fac}}\max\p{1, \sqrt{n/m}}$ for a 
matrix $\Aavg \in [-\infty,1]^{n \times m}$ satisfying 
Assumption~\ref{ass:quantile}. Then, for 
$\fac = \Omega\p{\frac{1}{\alpha^3\beta\epsilon^4}\max\p{1,\frac{m}{n}}}$, we can 
recover a matrix $\M \in [0,1]^{n \times m}$ satisfying the following 
properties:
\begin{align}
\label{eq:l1} &\|\M_i\|_1 \leq 1 \text{ \ for each $i \in \good$.} \\
\label{eq:linf} &\|\M_i\|_{\infty} \leq \frac{1}{\beta m}\|\M_i\|_1 \text{ \ for each $i \in \good$.} \\
\label{eq:close} &\frac{1}{|\good|}\sum_{i \in \good}\Big(\frac{1}{\beta m}\sum_{j \in Q_{\beta}(\Aavg_i)} \Aavg_{ij} - \sum_{j \in [m]} \M_{ij}\Aavg_{ij}\Big) \leq \error.
\end{align}
\end{theorem}
Note that properties \eqref{eq:l1} and \eqref{eq:linf} guarantee that $\M_i$ is a 
(sub-)probability distribution that is spread out over at least $\beta m$ 
elements, while property \eqref{eq:close} implies that the $\M_i$ are close to 
the uniform distribution over $Q_{\beta}(\Aavg_i)$ on average.

We next state our result for recovering $\rtrue$ from $\M$.
\begin{theorem}
\label{thm:main-r}
Suppose that $\M \in [0,1]^{n \times m}$ satisfies properties 
(\ref{eq:l1}-\ref{eq:close}) above, for an $\Aavg$ satisfying 
Assumption~\ref{ass:lipschitz}, and that $\robs$ satisfies Assumption~\ref{ass:r} 
relative to $\rtrue$. Then, for 
$\fac_0 = \Omega\p{\frac{\log(1/\alpha\failprob)}{\alpha\beta\error^2}}$, 
it is possible to recover a set $T$ satisfying
\[ \frac{1}{\beta m}\sum_{j \in Q_{\beta}(\rtrue)} \rtrue_j \leq \frac{1}{\beta m} \sum_{j \in T} \rtrue_j + 4\error. \]
\end{theorem}
Thus, we first recover $\M$ from $\Aobs$ using Theorem~\ref{thm:main-M}, and 
then recover $\rtrue$ from $\M$ using Theorem~\ref{thm:main-r}. We will see 
in more detail how to do so in Section~\ref{sec:approach}.
