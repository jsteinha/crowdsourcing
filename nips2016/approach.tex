\section{Approach}
\label{sec:approach}

As mentioned in Section~\ref{sec:assumptions}, our approach can be split into 
two parts: recovering a matrix $\M$ that approximates the $\beta$-quantiles of 
the raters, and using $\M$ to recover the $\beta$-quantile of $\rtrue$. We 
describe these respective components in Sections~\ref{sec:approach-M} and 
\ref{sec:approach-r} below.

\subsection{Recovering $\M$}
\label{sec:approach-M}
Given an $\Aobs$ with $\|\Aobs - \Aavg\|_{\op} \leq \facbound$, for an 
$\Aavg$ satisfying Assumption~\ref{ass:quantile}, we solve the following 
nuclear-norm optimization problem:

\begin{align}
\label{eq:optimization-noisy}
\text{maximize } &\langle \A, M \rangle, \\
\notag \text{ subject to } &M_{ij} \geq 0 \,\,\, \forall i,j, 
  &&\hskip-0.4in\sum_j M_{ij} \leq 1 \,\,\, \forall i, \\
\notag  &\max_j M_{ij} \leq \frac{1}{\beta m}\sum_j M_{ij} \,\,\, \forall i, 
  &&\hskip-0.4in\|M\|_* \leq \frac{2}{\alpha\epsilon}\sqrt{\frac{\alpha n}{\beta m}}, \phantom{xxxxxxx}
\end{align}
where $\|\cdot\|_*$ denotes the nuclear norm. In the sequel, 
we use $\sP$ to denote the constraint set in (\ref{eq:optimization-noisy}).

To gain some intuition, we will 
discuss each of the terms in \eqref{eq:optimization-noisy}. The objective 
$\langle \Aobs, M \rangle$ asks that $M$ match the noisy ratings 
$\Aobs$, while the nuclear norm constraint acts as a regularizer which will 
ensure that matching the noisy ratings $\Aobs$ implies that we also match the true 
ratings $\Aavg$. The constraints $M_{ij} \geq 0$, $\|M_i\|_1 \leq 1$, 
$\|M_i\|_{\infty} \leq \frac{1}{\beta m} \|M_i\|_1$ mean that each row of 
$M$ is a probability distribution supported on at least $\beta m$ elements. 

Constraining each row in $\ell^1$-norm is important in the adversarial setting, 
and contrasts with e.g. \citet{todo}, who instead constrain each row in 
$\ell^{\infty}$-norm. The $\ell^1$ constraint ensures that each row gets the 
same amount of ``weight'', and prevents adversarial strategies based on assigning 
very high ratings to a random $50\%$ of the items, which would otherwise 
increase the amount of work necessary by polynomial factors in $\frac{1}{\beta}$.

\paragraph{Analysis strategy.}
At a high level our strategy follows that of several other authors 
\citep{todo}, and proceeds by (1) comparing \eqref{eq:optimization-noisy} to a 
de-noised version, and (2) showing that the de-noised program recovers the 
information we want. However, the adversarial setting necessitates some 
differences: we only de-noise the part of $\Aobs$ corresponding to $\good$, 
and analyzing the de-noised program (which is usually straightforward) requires 
some care to ensure that the adversary cannot compromise the solution.

To make our approach more precise, let $\Aa$ be the matrix that is equal to 
$\Aavg_{\good}$ for the rows in $\good$, 
and equal to $\Aobs_{\bad}$ for the rows in $\bad$. Also let 
$R_{ij} = \frac{1}{\beta m}\bI[i \in \good, j \in Q_{\beta}]$. 
Ideally, we would like to have $M_{\good} = R_{\good}$, i.e. $M$ matches $R$ on 
all the rows of $\good$. In light of this, 
we will let $\Mm$ be the solution to the following ``corrected'' program, which 
we don't have access to (since it involves knowledge of $\Aavg$ and $\good$), 
but which will be useful for analysis purposes:
%\begin{align}
%\label{eq:optimization-noiseless}
%\text{maximize } &\langle \Aa, M \rangle, \\
%\notag \text{ subject to } &M_{\good} = R_{\good}, \\
%\notag  &M_{ij} \geq 0 \,\,\, \forall i,j, \\
%\notag  &\sum_j M_{ij} \leq 1 \,\,\, \forall i, \\
%\notag  &\max_j M_{ij} \leq \frac{1}{\beta m}\sum_j M_{ij} \,\,\, \forall i, \\
%\notag  &\|M\|_* \leq \frac{2}{\alpha\epsilon}\sqrt{\frac{\alpha n}{\beta m}}.
%\end{align}
\begin{align}
\label{eq:optimization-noiseless}
\text{maximize } &\langle \Aa, M \rangle, \\
\notag \text{ subject to } &M_{\good} = R_{\good}, \\
\notag  &M_{ij} \geq 0 \,\,\, \forall i,j, 
  &&\hskip-0.4in\sum_j M_{ij} \leq 1 \,\,\, \forall i, \\
\notag  &\max_j M_{ij} \leq \frac{1}{\beta m}\sum_j M_{ij} \,\,\, \forall i, 
  &&\hskip-0.4in\|M\|_* \leq \frac{2}{\alpha\epsilon}\sqrt{\frac{\alpha n}{\beta m}}, \phantom{xxxxxxx}
\end{align}

Our goal is to show that $\M$ is ``close'' to $M^*$ and hence gives information 
about $\good$ and $Q_{\beta}$. 
We formalize this in the following theorem:
\begin{theorem}
\label{thm:main}
Let $\fac = \Omega\p{\frac{1}{\alpha^3\beta\epsilon^4}\max\p{\frac{m}{n},1}}$. Then, it is the case that 
\[ \frac{1}{|\good|} \sum_{i \in \good} \p{\frac{1}{\beta m} \sum_{j \in Q_{\beta}(\Aavg_i)} \Aavg_{ij} - \sum_{j \in [m]} \M_{ij} \Aavg_{ij}} \leq \epsilon. \]
%$\|\M_i - \pi^*\|_1 \leq \epsilon$, 
%where $\pi^*_j = \frac{1}{|\sD|}$ if $j \in \sD$, and $0$ otherwise.
\end{theorem}
Theorem~\ref{thm:main} says that on average across the good rows, we recover 
a probability distribution whose expected rating is within $\epsilon$ of 
the rating of the $\beta$-quantile. 

The two key components in proving Theorem~\ref{thm:main} are 
given as Proposition~\ref{prop:objective-bound} and 
Proposition~\ref{prop:subgradient} below. First, we show 
(Proposition~\ref{prop:objective-bound}) that $\M$ is an approximate optimizer 
of \ref{eq:optimization-noiseless}, which is true because 
$\langle \Aobs, M \rangle$ and $\langle \Aa, M \rangle$ are close together for 
all $M \in \sP$.
\begin{proposition}
\label{prop:objective-bound}
Let $\fac = \Omega\p{\frac{1}{\alpha^3\beta\epsilon^4}\max\p{\frac{m}{n},1}}$. For $\M$ and $\Mm$ as defined above, we have
\begin{equation}
\label{eq:objective-bound}
\langle \Aa, \M \rangle \geq \langle \Aa, \Mm \rangle - \epsilon \alpha n.
\end{equation}
\end{proposition}

Next, we need to show that the nuclear norm constraint does 
not create too much bias in the de-noised program 
\eqref{eq:optimization-noiseless}. Note that this is the \emph{only} constraint 
that affects more than one row in \eqref{eq:optimization-noiseless}, and without 
it it would be apparent that every $M^*_i$, for $i \in \good$, would be equal to the 
$\beta$-quantile $Q_{\beta}$. Our goal is to show that the nuclear norm 
constraint doesn't create too much dependence between the rows, which we can 
formalize by bounding the Lagrangian of \eqref{eq:optimization-noiseless}. 
Doing so leads to:

\begin{proposition}
\label{prop:subgradient}
Let $\fac = \Omega\p{\frac{1}{\alpha\beta\epsilon^2}\frac{m}{n}}$. Then there exists a matrix $Z$ with 
$\rank(Z) = 1$, $\|Z\|_F \leq \epsilon \sqrt{\alpha\beta mn}$ such that
\begin{align}
\langle \Aa_{\good} - Z_{\good}, \Mm_{\good} - M_{\good} \rangle &\leq \langle \Aa, \Mm - M \rangle \text{ for all $M \in \sP$}.
\end{align}
\end{proposition}
Essentially, Proposition~\ref{prop:subgradient} shows that any change in 
$\langle \Aa, M \rangle$ caused by changing $M_{\bad}$ can be upper-bounded 
by a small term $\langle Z_{\good}, \Mm_{\good} - M_{\good} \rangle$ that depends only 
on $M_{\good}$, thereby bounding the effect that the adversaries can have 
on $M_{\good}$. Proposition~\ref{prop:subgradient} is therefore the key 
technical tool powering our results, and we devote 
Section~\ref{sec:subgradient-proof} to its proof. (Proposition~\ref{prop:objective-bound} is proved in Appendix~\ref{sec:objective-bound-proof}.)

In the remainder of this section, we will prove Theorem~\ref{thm:main} from Propositions~\ref{prop:objective-bound} 
and \ref{prop:subgradient}.
\begin{proof}[Proof of Theorem~\ref{thm:main}]
We start by plugging in $\M$ for $M$ in Proposition~\ref{prop:subgradient}. This yields
\begin{align}
\langle \Aa_{\good} + Z_{\good}, \Mm_{\good} - \M_{\good} \rangle &\leq \langle \Aa, \Mm - \M \rangle 
 \leq \epsilon \alpha n
\end{align}
by Proposition~\ref{prop:objective-bound}.
On the other hand, we have 
\begin{align}
|\langle Z_{\good}, \Mm_{\good} - \M_{\good} \rangle| &\leq \|Z_{\good}\|_F\|\Mm_{\good} - \M_{\good}\|_F \\
 &\leq \epsilon\sqrt{\alpha\beta nm} \sqrt{\|\Mm_{\good} - \M_{\good}\|_{1}\|\Mm_{\good} - \M_{\good}\|_{\infty}} \\
 &\leq \epsilon\sqrt{\alpha\beta nm} \sqrt{2\alpha n/\beta m} = \sqrt{2}\epsilon\alpha n.
\end{align}
Putting these together, we obtain
\begin{align}
\langle \Aa_{\good}, \Mm_{\good} - \M_{\good} \rangle &\leq (1+\sqrt{2})\epsilon \alpha n.
\end{align}
Expanding $\langle \Aa_{\good}, \Mm_{\good} - \M_{\good} \rangle$ as 
$\sum_{i \in \good}\p{\sum_{j \in [m]} (R_{ij} - \M_{ij})\Aavg_{ij}}$,
we obtain 
\[ \frac{1}{|\good|}\sum_{i \in \good}\Big(\frac{1}{\beta m}\sum_{j \in Q_{\beta}} \Aavg_{ij} - \sum_{j \in [m]} \M_{ij}\Aavg_{ij}\Big) \leq (1+\sqrt{2})\epsilon. \]
The desired result then follows after scaling down $\epsilon$ 
by a factor of $1+\sqrt{2}$.
\end{proof}

\subsection{Recovering $Q_{\beta}(\rtrue)$}
\label{sec:approach-r}

The algorithm for recovering $Q_{\beta}(\rtrue)$ from $\M$ and $\robs$ 
is simple, and involves two steps:
\begin{enumerate}
\item Greedily choose the $\goodfrac n$ rows of $\M$ (call then $\goodobs$) 
      for which $\sum_{j \in [m]} \M_{ij}\robs_j$ is largest.
\item Let $\rdist \eqdef \frac{1}{\goodfrac n} \sum_{i \in \goodobs} \M_i$, 
      and choose $T$ to be the $\beta m$ coordinates for which $\rdist_j$ is 
      largest.
\end{enumerate}
\todo{provide result and analysis}

%%% TODO --- this is good intuition, should include it somewhere
%%% At a high level, our strategy is the following --- since $\Mm$ is already optimized 
%%% on the rows outside of $\good$, we should (``morally'', though not in actuality) have 
%%% $\langle \Aa_{\neg \good}, \Mm_{\neg \good} \rangle \geq \langle \Aa_{\neg \good}, \M_{\neg \good} \rangle$. 
%%% 
%%% Looking back at the rows inside $\good$, we must therefore have 
%%% $\langle \Aa_{\good}, \M_{\good} \rangle \geq \langle \Aa_{\good}, \Mm_{\good} \rangle - \epsilon \alpha n$, 
%%% which is exactly the statement \eqref{eq:M-bound} with $(1+\sqrt{2})\epsilon$ replaced by $\epsilon$.
%%% 
%%% The problem with the above argument is that, even though $\Mm$ is optimized on $\neg \good$, the rows in 
%%% $\good$ and $\neg \good$ are coupled by the joint constraint $\|M\|_* \leq \frac{2}{\epsilon\alpha}$. We 
%%% will decouple the rows by taking the subgradient of the Lagrangian at $\Mm$, which yields an upper bound 
%%% for our optimization problem. By doing this we obtain:

