\documentclass[anon,12pt]{colt2016} % Anonymized submission
\usepackage{amsmath,amsfonts,amssymb}
\usepackage{hyperref,url}
\usepackage{import,subfiles}
%\usepackage{tabularx}
%\usepackage{algorithm}
%\usepackage{algorithmic}
\usepackage{multirow}
%\usepackage{hyperref,url}
%\usepackage{amsmath,amssymb}%,amsthm}
%\usepackage{tikz}
%\usepackage{xspace}
%\usepackage{stmaryrd,wasysym}%,clrscode}
%\usepackage{etex,etoolbox}
%\usepackage{ifthen}
%\usetikzlibrary{patterns,positioning}

%%% Added by Jonathan %%%
\def\[#1\]{\begin{align}#1\end{align}}
\def\(#1\){\begin{align*}#1\end{align*}}
\newcommand{\ip}[2]{\left\langle #1, #2 \right\rangle}
\newcommand{\dee}{\mathrm{d}}
\def\argmax{\operatornamewithlimits{arg\,max}}
\def\argmin{\operatornamewithlimits{arg\,min}}
\newcommand{\defined}{\ensuremath{\triangleq}}
\newcommand{\bprf}{\begin{proof}}
\newcommand{\eprf}{\end{proof}}
\newcommand{\blem}{\begin{lemma}}
\newcommand{\elem}{\end{lemma}}
\newcommand{\eps}{\epsilon}
%%% End added by Jonathan %%%

\DeclareMathOperator{\Uniform}{Uniform}
\DeclareMathOperator{\polylog}{polylog}
\newcommand{\oo}{\mathcal{O}}
%\newcommand{\p}[1]{\left(#1\right)}
\DeclareMathOperator{\Geometric}{Geometric}
\newcommand{\op}{\mathrm{op}}
\newcommand{\Beta}{B}
%\DeclareMathOperator{\diag}{diag}
\DeclareMathOperator{\Regret}{Regret}
\newcommand{\algname}[1]{\textsc{\lowercase{#1}}\xspace}
\newcommand{\bmd}{\algname{BMD}}
\newcommand{\primal}{\algname{PBMD}}
\newcommand{\dual}{\algname{DBMD}}
\newcommand{\cgd}{\algname{CGD}}
\newcommand{\md}{\algname{MD}}

\newcommand{\lc}[1]{#1_{\mathrm{loc}}}
\newcommand{\eq}[1]{\stackrel{\mathrm{#1}}{=}}
\DeclareMathOperator{\Var}{Var}
%\DeclareMathOperator{\sign}{sign}
\DeclareMathOperator{\MMD}{MMD}
\newcommand{\MMDr}{\tilde{\MMD}}
%\DeclareMathOperator{\tr}{tr}
%\newcommand{\inner}[2]{\langle #1, #2 \rangle}
%\newcommand{\E}{\mathcal{E}}
%\newcommand{\eqdef}{\stackrel{\mathrm{def}}{=}}
%\newcommand{\bP}{\mathbb{P}}
%\newcommand{\bI}{\mathbb{I}}
%\newcommand{\bE}{\mathbb{E}}
%\newcommand{\sF}{\mathcal{F}}
%\newcommand{\sV}{\mathcal{V}}
%\newcommand{\sH}{\mathcal{H}}
%\newcommand{\sK}{\mathcal{K}}
%\newcommand{\sA}{\mathcal{A}}
%\newcommand{\sC}{\mathcal{C}}
%\newcommand{\sM}{\mathcal{M}}
%\newcommand{\sE}{\mathcal{E}}
%\newcommand{\C}{\mathcal{C}}
%\newcommand{\sB}{\mathcal{B}}
%\newcommand{\bR}{\mathbb{R}}
%\newcommand{\bN}{\mathbb{N}}
%\newcommand{\bZ}{\mathbb{Z}}
%\newcommand{\bF}{\mathbb{F}}
%\newcommand{\sI}{\mathcal{I}}
%\newcommand{\sP}{\mathcal{P}}
%\newcommand{\sX}{\mathcal{X}}
%\newcommand{\sY}{\mathcal{Y}}
%\newcommand{\sZ}{\mathcal{Z}}
%\newcommand{\sS}{\mathcal{S}}
%\newcommand{\sJ}{\mathcal{J}}
%\newcommand{\sR}{\mathcal{R}}
%\newcommand{\sN}{\mathcal{N}}
\newcommand{\meet}{\wedge}
\newcommand{\RE}[2]{\operatorname{RE}\left(#1 \ \| \ #2\right)}
%\newcommand{\KL}[2]{\operatorname{KL}\left(#1 \ \| \ #2\right)}
\newcommand{\KLm}[2]{\operatorname{KL}_m\left(#1 \ \| \ #2\right)}
\newcommand{\score}[2]{\operatorname{score}\left(#1 \ \| \ #2\right)}
\newcommand{\phih}{\hat{\phi}}
\newcommand{\psih}{\hat{\psi}}
\DeclareMathOperator{\supp}{supp}
\DeclareMathOperator{\loc}{loc}
\DeclareMathOperator{\lub}{lub}
\newcommand{\atom}[1]{#1^{\circ}}
\newcommand{\stitch}[2]{\overline{#1}^{#2}}
%\DeclareMathOperator{\argmin}{argmin}
%\DeclareMathOperator{\argmax}{argmax}

%\newtheorem{theorem}{Theorem}[section]
%\newtheorem{lemma}[theorem]{Lemma}
%\newtheorem{proposition}[theorem]{Proposition}
%\newtheorem{corollary}[theorem]{Corollary}
%\newtheorem{assumption}[theorem]{Assumption}
%\theoremstyle{definition}
%\newtheorem{example}[theorem]{Example}
%\newtheorem{definition}[theorem]{Definition}
%\newtheorem{remark}[theorem]{Remark}
%\newtheorem{property}[theorem]{Property}

\def\ci{\perp\!\!\!\perp}

\usepackage{breqn}

% \documentclass{colt2016} % Include author names

% The following packages will be automatically loaded:
% amsmath, amssymb, natbib, graphicx, url, algorithm2e

\title[Adversarial Crowdsourcing without Ground Truth Labels]{Adversarial Crowdsourcing without Ground Truth Labels}
\usepackage{times}
 % Use \Name{Author Name} to specify the name.
 % If the surname contains spaces, enclose the surname
 % in braces, e.g. \Name{John {Smith Jones}} similarly
 % if the name has a "von" part, e.g \Name{Jane {de Winter}}.
 % If the first letter in the forenames is a diacritic
 % enclose the diacritic in braces, e.g. \Name{{\'E}louise Smith}

 % Two authors with the same address
  % \coltauthor{\Name{Author Name1} \Email{abc@sample.com}\and
  %  \Name{Author Name2} \Email{xyz@sample.com}\\
  %  \addr Address}

 % Three or more authors with the same address:
 % \coltauthor{\Name{Author Name1} \Email{an1@sample.com}\\
 %  \Name{Author Name2} \Email{an2@sample.com}\\
 %  \Name{Author Name3} \Email{an3@sample.com}\\
 %  \addr Address}


 % Authors with different addresses:
 \coltauthor{\Name{Jacob Steinhardt} \Email{jsteinhardt@cs.stanford.edu}\\
 \addr 353 Serra Mall, Stanford, CA 94305
 \AND
 \Name{Gregory Valiant} \Email{valiant@stanford.edu}\\
 \addr 353 Serra Mall, Stanford, CA 94305
 }

\newcommand{\goodfrac}{\alpha}
\newcommand{\quality}{\epsilon}
\newcommand{\failprob}{\delta}
\begin{document}

\maketitle

\begin{abstract}
We consider a crowdsourcing model in which a fraction $\goodfrac$ of 
workers can both generate high-quality data and correctly evaluate 
the quality of data, while the remaining workers may behave adversarially. 
There is no ground truth on which to evaluate worker quality, but the 
manager of the experiment can manually evaluate the quality of 
a datum after the fact. Given $N$ workers, we wish to curate a dataset 
of size $\Omega(\goodfrac N)$, with at most an $\quality$ fraction of low-quality 
data. We show that this is possible with 
$\oo\p{\frac{\log(1/\quality\goodfrac)}{\goodfrac\min\p{\quality,\goodfrac}}}$ 
work per worker, and $\oo\p{\frac{\log(1/\goodfrac)}{\goodfrac\quality}}$ 
work for the manager.
\end{abstract}

\begin{keywords}
\end{keywords}

\section{Introduction}

While several mechanisms have been proposed for ensuring high-quality 
data in crowdsourcing applications, most assume that there are ``gold labels'' 
that can be used to assess worker performance during the course of 
data collection; moreover, most mechanisms make assumptions about the behavior 
of workers (e.g. that workers that behaved well before will continue to 
behave well, or that they act to maximize a certain payoff function).
In this work we are motivated by cases where there is not necessarily a single 
correct answer --- for instance, tasks such as ``draw an interesting picture'' 
or ``translate this sentence'', where different workers could generate different 
equally good answers. In this case, the only way to evaluate quality is 
via evaluation by other workers or by the manager of the experiment. 
Moreover, since workers will often try to game the system (in order to 
get paid), we would like to model non-cooperative workers as adversarial. 
To formalize both of these desiderata, we adopt the following model:
\begin{itemize}
\item There is a manager who controls the experiment, and 
      $N$ workers.
\item The goal is to generate data $X \in \sX$ that satisfies 
      a binary criterion $f : \sX \to \{0,1\}$ (e.g. $f$ could answer
      ``Is $X$ an interesting picture?'' or ``Is $X$ a correct translation?'').
      The manager knows $f$.
\item Each worker $i$ can generate a datum $X_i$, as well as evaluate 
      data with a function $f_i : \sX \to \{0,1\}$.
\item There are $\goodfrac N$ \emph{cooperative} workers, such that 
      $f_i = f$, $f(X_i) = 1$, and the worker 
      wants to be helpful (they will cooperate with the protocol 
      specified by the manager).
\item The rest of the workers could be adversarial; they could generate 
      good data or bad data, and could output correct or incorrect evaluations. 
      The adversaries could have knowledge of who they are.
\end{itemize}
Our goal is to curate as much good data as possible, while letting very little 
bad data in. In particular, after allowing both the manager and the workers to 
make some number of evaluations, we wish to output a dataset 
$\{X_{i_1}, \ldots, X_{i_m}\}$ such that 
\begin{equation}
\label{eq:target}
m = \Omega(\delta N)\text{ and }\frac{1}{m} \sum_{j=1}^m f(X_{i_j}) \geq 1-\quality.
\end{equation}
Note that $m = \Omega(\delta N)$ asks that our dataset be within a constant 
factor of the number of good workers. We could imagine adding this constant 
factor as another parameter in the problem, but avoid this for now for simplicity.

Our main result is the following:
\begin{theorem}
\label{thm:main}
For any $\failprob$, there is a protocol that requires 
$\oo\p{\frac{\log(1/\quality\goodfrac\failprob)}{\goodfrac\min\p{\quality,\goodfrac}}}$ evaluations from each worker, and $\oo\p{\frac{\log(1/\goodfrac\failprob)}{\goodfrac\quality}}$ evaluations from the manager, such that the output dataset 
satisfies \eqref{eq:target} with probability $1-\failprob$ and with $m \geq \frac{1}{2}\delta N$.
\end{theorem}
The proof is not long, though we note that na\"ive approaches will 
accrue additional $\polylog(N)$ factors on the number of evaluations per worker.
We think the main aspects of interest in this work are the setting itself 
(which seems natural but relatively un-studied) as well as the ability to do 
average work that is independent of $N$.

The idea behind the protocol is the following: after each worker generates their 
datum, we will have each worker evaluate some number $k$ of other workers at 
random, and draw directed edges labeled with either $+$ (for a positive 
evaluation) or $-$ (for a negative evaluation). This is illustrated in 
Figure~\ref{fig:illustration}. We then search for a large \emph{community}, 
defined as a set of workers where all edges within the community are positive. 
We can show that any such community must have either high average quality 
or very few cooperative workers:
\begin{lemma}
\label{lem:community}
Suppose that each worker evaluates $k = C \cdot \frac{\log(1/\quality\goodfrac\failprob)}{\goodfrac\min(\quality,\goodfrac)}$ 
other workers for some universal constant $C$. 
Then, with probability $1-\delta$, every 
community must have either at most $\frac{1}{2}\quality\goodfrac N$ bad data 
or at most $\frac{1}{4}\goodfrac^2 N$ cooperative workers.
\end{lemma}
We also note that the good workers themselves form a community of size 
$\goodfrac N$. Therefore, we can perform the following algorithm, which loops 
until eventually accepting.
\begin{itemize}
\item Find a community of size at least $\frac{1}{2}\goodfrac N$.
\item Sample roughly $\frac{1}{\quality}$ data at random from the community, 
      and check whether the average quality is at least $1-\quality$.
\item If it is, use all the data in the community as the dataset, 
      otherwise remove all members of the community from consideration 
      and continue.
\end{itemize}
This algorithm is given more formally as Algorithm~\ref{alg:algorithm}. 
Lemma~\ref{lem:community} ensures that, whenever we remove a community 
from consideration, we are removing only a small number of cooperative 
workers; in particular, we remove at most $\frac{1}{4}\goodfrac^2 N$ cooperative 
workers each time, while Algorithm~\ref{alg:algorithm} can loop at most 
$\frac{2}{\goodfrac}$ times before terminating; therefore, at most 
$\frac{1}{2}\goodfrac^2 N$ cooperative workers are removed in total, and 
so we will always have enough cooperative workers left to form a large 
community with good data.

The remainder of this paper is devoted to proving Theorem~\ref{thm:main}. 
In Section~\ref{sec:community-proof}, we prove Lemma~\ref{lem:community}, 
while in Section~\ref{sec:main-proof} we use Lemma~\ref{lem:community} 
to prove Theorem~\ref{thm:main}.

\section{Proof of Lemma~\ref{lem:community}}
\label{sec:community-proof}

\newcommand{\nbad}{n_{\mathrm{bad}}}
\newcommand{\ncoop}{n_{\mathrm{coop}}}

Our proof proceeds via a union bound. 
We will show that for every set of $\nbad = \frac{1}{2}\quality\goodfrac N$ 
workers with bad data, and every set of $\ncoop = \frac{1}{4}\goodfrac^2 N$ 
cooperative workers, the probability that there are no negative 
edges is very low --- in particular, low enough that we can 
union bound over all $\binom{N}{\nbad}\binom{N}{\ncoop}$ possible 
pairs of such sets. To show this, note that any edge from a cooperative 
worker to a bad worker would be negative, so we have no negative edges only 
if there are no edges at all. If each worker evaluates $k$ other workers at 
random, then the probability that this occurs is 
\begin{equation}
\label{eq:single-prob}
(1-\nbad/N)^{k\ncoop} \leq \exp\p{-\frac{1}{8}k\quality\goodfrac^3N}. 
\end{equation}
The number of sets we need to union bound over is 
\begin{align}\binom{N}{\nbad}\binom{N}{\ncoop} &\leq \p{\frac{2e}{\quality\goodfrac}}^{\nbad}\p{\frac{4e}{\goodfrac^2}}^{\ncoop} \\
\label{eq:union} &\leq \exp\p{\frac{1}{2}\quality\goodfrac N\log\p{\frac{2e}{\quality\goodfrac}} + \frac{1}{4}\goodfrac^2 N\log\p{\frac{4e}{\goodfrac^2}}}.
\end{align}
We need the product of \eqref{eq:single-prob} with \eqref{eq:union} to be at most 
$\delta$, which is satisfied if 
$k \geq 8\p{\frac{\log\p{\frac{2e}{\quality\goodfrac}}}{\goodfrac\min\p{\quality,\goodfrac}} + \frac{\log\p{\frac{1}{\delta}}}{\quality\goodfrac^3N}}$,
thus proving Lemma~\ref{lem:community}.

\section{Proof of Theorem~\ref{thm:main}}
\label{sec:main-proof}

\acks{}

\bibliography{refdb/all.bib}

%\appendix
%
%\section{My Proof of Theorem 1}
%
%This is a boring technical proof.
%
%\section{My Proof of Theorem 2}
%
%This is a complete version of a proof sketched in the main text.

\end{document}
