\documentclass[11pt]{article}
\usepackage{amsmath,amsfonts,amsthm,amssymb}
\usepackage{hyperref,url}
\usepackage{import,subfiles}
\usepackage{tabularx}
\usepackage{bbm}
\usepackage{algorithm}
\usepackage{algorithmic}
\usepackage{multirow}
\usepackage{fullpage}
\usepackage{natbib}
%\usepackage{hyperref,url}
%\usepackage{amsmath,amssymb}%,amsthm}
%\usepackage{tikz}
%\usepackage{xspace}
%\usepackage{stmaryrd,wasysym}%,clrscode}
%\usepackage{etex,etoolbox}
%\usepackage{ifthen}
%\usetikzlibrary{patterns,positioning}

%%% Added by Jonathan %%%
\def\[#1\]{\begin{align}#1\end{align}}
\def\(#1\){\begin{align*}#1\end{align*}}
\newcommand{\ip}[2]{\left\langle #1, #2 \right\rangle}
\newcommand{\dee}{\mathrm{d}}
\def\argmax{\operatornamewithlimits{arg\,max}}
\def\argmin{\operatornamewithlimits{arg\,min}}
\newcommand{\defined}{\ensuremath{\triangleq}}
\newcommand{\bprf}{\begin{proof}}
\newcommand{\eprf}{\end{proof}}
\newcommand{\blem}{\begin{lemma}}
\newcommand{\elem}{\end{lemma}}
\newcommand{\eps}{\epsilon}
%%% End added by Jonathan %%%

\DeclareMathOperator{\Uniform}{Uniform}
\DeclareMathOperator{\polylog}{polylog}
\newcommand{\oo}{\mathcal{O}}
%\newcommand{\p}[1]{\left(#1\right)}
\DeclareMathOperator{\Geometric}{Geometric}
\newcommand{\op}{\mathrm{op}}
\newcommand{\Beta}{B}
%\DeclareMathOperator{\diag}{diag}
\DeclareMathOperator{\Regret}{Regret}
\newcommand{\algname}[1]{\textsc{\lowercase{#1}}\xspace}
\newcommand{\bmd}{\algname{BMD}}
\newcommand{\primal}{\algname{PBMD}}
\newcommand{\dual}{\algname{DBMD}}
\newcommand{\cgd}{\algname{CGD}}
\newcommand{\md}{\algname{MD}}

\newcommand{\lc}[1]{#1_{\mathrm{loc}}}
\newcommand{\eq}[1]{\stackrel{\mathrm{#1}}{=}}
\DeclareMathOperator{\Var}{Var}
%\DeclareMathOperator{\sign}{sign}
\DeclareMathOperator{\MMD}{MMD}
\newcommand{\MMDr}{\tilde{\MMD}}
%\DeclareMathOperator{\tr}{tr}
%\newcommand{\inner}[2]{\langle #1, #2 \rangle}
%\newcommand{\E}{\mathcal{E}}
%\newcommand{\eqdef}{\stackrel{\mathrm{def}}{=}}
%\newcommand{\bP}{\mathbb{P}}
%\newcommand{\bI}{\mathbb{I}}
%\newcommand{\bE}{\mathbb{E}}
%\newcommand{\sF}{\mathcal{F}}
%\newcommand{\sV}{\mathcal{V}}
%\newcommand{\sH}{\mathcal{H}}
%\newcommand{\sK}{\mathcal{K}}
%\newcommand{\sA}{\mathcal{A}}
%\newcommand{\sC}{\mathcal{C}}
%\newcommand{\sM}{\mathcal{M}}
%\newcommand{\sE}{\mathcal{E}}
%\newcommand{\C}{\mathcal{C}}
%\newcommand{\sB}{\mathcal{B}}
%\newcommand{\bR}{\mathbb{R}}
%\newcommand{\bN}{\mathbb{N}}
%\newcommand{\bZ}{\mathbb{Z}}
%\newcommand{\bF}{\mathbb{F}}
%\newcommand{\sI}{\mathcal{I}}
%\newcommand{\sP}{\mathcal{P}}
%\newcommand{\sX}{\mathcal{X}}
%\newcommand{\sY}{\mathcal{Y}}
%\newcommand{\sZ}{\mathcal{Z}}
%\newcommand{\sS}{\mathcal{S}}
%\newcommand{\sJ}{\mathcal{J}}
%\newcommand{\sR}{\mathcal{R}}
%\newcommand{\sN}{\mathcal{N}}
\newcommand{\meet}{\wedge}
\newcommand{\RE}[2]{\operatorname{RE}\left(#1 \ \| \ #2\right)}
%\newcommand{\KL}[2]{\operatorname{KL}\left(#1 \ \| \ #2\right)}
\newcommand{\KLm}[2]{\operatorname{KL}_m\left(#1 \ \| \ #2\right)}
\newcommand{\score}[2]{\operatorname{score}\left(#1 \ \| \ #2\right)}
\newcommand{\phih}{\hat{\phi}}
\newcommand{\psih}{\hat{\psi}}
\DeclareMathOperator{\supp}{supp}
\DeclareMathOperator{\loc}{loc}
\DeclareMathOperator{\lub}{lub}
\newcommand{\atom}[1]{#1^{\circ}}
\newcommand{\stitch}[2]{\overline{#1}^{#2}}
%\DeclareMathOperator{\argmin}{argmin}
%\DeclareMathOperator{\argmax}{argmax}

%\newtheorem{theorem}{Theorem}[section]
%\newtheorem{lemma}[theorem]{Lemma}
%\newtheorem{proposition}[theorem]{Proposition}
%\newtheorem{corollary}[theorem]{Corollary}
%\newtheorem{assumption}[theorem]{Assumption}
%\theoremstyle{definition}
%\newtheorem{example}[theorem]{Example}
%\newtheorem{definition}[theorem]{Definition}
%\newtheorem{remark}[theorem]{Remark}
%\newtheorem{property}[theorem]{Property}

\def\ci{\perp\!\!\!\perp}

\definecolor{mydarkblue}{rgb}{0,0.08,0.45}
\hypersetup{ %
  pdftitle={},
  pdfauthor={},
  pdfsubject={Proceedings of the International Conference on Machine Learning 2016},
  pdfkeywords={},
  pdfborder=0 0 0,
  pdfpagemode=UseNone,
  colorlinks=true,
  linkcolor=mydarkblue,
  citecolor=mydarkblue,
  filecolor=mydarkblue,
  urlcolor=mydarkblue,
  pdfview=FitH}
\setcitestyle{authoryear,round,citesep={;},aysep={,},yysep={;}}
\usepackage{breqn}
\DeclareMathOperator{\Tr}{Tr}
\newcommand{\M}{\tilde{M}}
\newcommand{\A}{\hat{A}}
\DeclareMathOperator{\poly}{poly}
\newcommand{\sD}{\mathcal{D}}
\newcommand{\oo}{\mathcal{O}}
\newcommand{\bi}{\mathbbm{1}}
\newcommand{\todo}[1]{{\color{red} [TODO: {#1}]}}
\DeclareMathOperator{\round}{round}

\begin{document}

\section{Achieving $\poly(\alpha^{-1})$ degree}
We assume we have a matrix $A \in \bR^{n \times n}$ and sets 
$\sC \subset [n], \sD \subset [n]$ satisfying the following:
\begin{itemize}
\item $A_{i,j} = +1$ for all $i \in \sC$, $j \in \sD$
\item $A_{i,j} = -1$ for all $i \in \sC$, $j \not\in \sD$
\end{itemize}
Moreover, assume that $\min(|\sC|, |\sD|) \geq \alpha n$.

Finally, we let $\A$ be a random matrix with i.i.d. entries such that $\bE[\A] = A$, 
and $\bE[(\A_{ij}-A_{ij})^{2p}] \leq \frac{(2p)!}{2^pp!} \p{n/d}^{p}$ for all $p \in \{1,2,\ldots\}$ 
and some fixed $d \in \bR_{>0}$. For instance, this holds if $\Var[\A_{ij}] \leq \frac{n}{d}$ and $|\A_{ij}| \leq \frac{n}{d}$ almost surely, 
which we can obtain by sampling each entry of $A$ with probability $\frac{d}{n}$ and scaling appropriately (which results in $d$ entries in 
each row in expectation).
\todo{the moment assumption doesn't actually hold in this case, and making it hold costs us a $\sqrt{\log(n)}$ factor}

Let $\M$ be the solution to the following program:
\begin{equation}
\label{eq:optimization-noisy}
\text{maximize } \langle \A, M \rangle - \mu \|M\|_*, \text{ subject to } 0 \leq M_{ij} \leq 1 \, \forall i,j,
\end{equation}
where $\|\cdot\|_*$ denotes the nuclear norm.

We also let $M^*$ be the solution to the ``noiseless'' program
\begin{equation}
\label{eq:optimization-noiseless}
\text{maximize } \langle A, M \rangle - \mu \|M\|_*, \text{ subject to } 0 \leq M_{ij} \leq 1 \, \forall i,j,
\end{equation}

Our goal is to show that $\M$ is ``close'' to giving correct information about $\sC$ and $\sD$. 
We will proceed in $2$ steps:
\begin{enumerate}
\item Show that $\M$ is an approximate optimizer of \eqref{eq:optimization-noiseless}.
%\item Characterize the optimum $M^*$ of \eqref{eq:optimization-noiseless}.
\item Show that any approximate optimizer of \eqref{eq:optimization-noiseless} is close to identifying $\sD$ for most rows in $\sC$.
\end{enumerate}
All relevant lemmas are proved in Section~\ref{sec:lemma-proofs}.

For the first step, we begin with the trivial observation that 
$\max\p{\|\M\|_*, \|M^*\|_*} \leq \frac{n^2}{\mu}$, since otherwise $M = 0$ 
would have a larger objective value for (\ref{eq:optimization-noisy},\ref{eq:optimization-noiseless}). 
We then have the following inequality:
\begin{lemma}
\label{lem:concentration}
With probability $1-\delta$, we have
\[ |\langle A-\A, M \rangle| \leq \oo\p{\frac{n^3}{\mu}\p{\frac{1}{\sqrt{d}} + \frac{1}{d}\sqrt{\log(1/\delta)}}} \]
for all $M$ satisfying $\|M\|_* \leq \frac{n^2}{\mu}$.
\end{lemma}
\paragraph{Analyzing $\M$.} Now, we are ready to show that 
$\M$ approximately optimizes \eqref{eq:optimization-noiseless}. 
We will take $\mu = \epsilon \alpha n$, 
$d = \Omega\p{\max\p{\frac{1}{(\alpha^3\epsilon^2)^2}, \frac{\sqrt{\log(1/\delta)}}{\alpha^3\epsilon^2}}}$, which yields (for appropriate constants)
\begin{align}
|\langle A-\A, M \rangle| &\leq \frac{1}{2}\epsilon \alpha^2 N^2.
\end{align}
We then have
\begin{align}
\langle A, \M \rangle - \mu \|\M\|_* &\geq \langle \A, \M \rangle - \mu \|\M\|_* - \frac{1}{2}\epsilon \alpha^2 N^2 \\
 &\geq \langle \A, M^* \rangle - \mu \|M^*\|_* - \frac{1}{2} \epsilon \alpha^2 N^2 \\
 &\geq \langle A, M^* \rangle - \mu \|M^*\|_* - \epsilon \alpha^2 N^2.
\end{align}
It follows that $\M$ is within $\epsilon \alpha^2N^2$ of the optimum of \eqref{eq:optimization-noiseless}, 
at least in terms of objective value.


\section{Bounding $M^* - \M$}
To start, we need some notation. Let $r(M)$ be the matrix such that
\[ r(M)_{ij} = \left\{ \begin{array}{ccl} \frac{1+A_{ij}}{2} & : & i \in \sC \\ M_{ij} & : & \text{else} \end{array} \right.. \]
In other words, $r$ replaces the rows in $\sC$ with their target values, and 
leaves the rest of the rows fixed.

We also define the projection matrix $P$ as
\[ P_{i,i'} = \left\{ \begin{array}{ccl} \frac{1}{|\sC|} & : & i, i' \in \sC \\ \delta_{i,i'} & : \text{else} \end{array} \right.. \]
Thus $PM = M$ if and only if all rows in $\sC$ are equal to each other.

We have the following technical lemma which controls 
the subgradient of the objective function:
\begin{lemma}
\label{lem:subgradient}
Let $f(M) = \langle A, M \rangle - \mu \|M\|_*$, and let 
$M_0$ satisfy $M_0 = PM_0$. Then,
there is a $Z_0 \in \partial f(M_0)$ such that, when restricted 
to the rows lying in $\sC$, we have
\[ (Z_0)_{\sC} = \bi \cdot \left[(2\bi_{\sD}-\bi)^{\top} - \mu v_0^{\top}\right], \]
for some vector $v_0$ of norm at most $\frac{1}{\sqrt{|\sC|}}$.
\end{lemma}

We are now ready to show that $\M$ is close to $M^*$. Indeed, we already 
have that $f(\M) \geq f(M^*) - \epsilon \alpha^2 N^2$. Letting $M' \eqdef r(\M)$ 
and re-arranging, we have
\begin{align}
\epsilon \alpha^2 N^2 &\geq f(M^*) - f(\M) \\
 &\geq f(M') - f(\M) \\
 &\geq \langle M' - \M, Z \rangle,
\end{align}
where $Z \in \partial f(M')$.
By Lemma~\ref{lem:subgradient}, we know that $Z_{\sC} = \bi \cdot \left[(2\bi_{\sD}-\bi)^{\top} - \mu v^{\top}\right]$, 
where $\|v\|_2^2 \leq \frac{1}{|\sC|}$. Since $M'_{ij} = 1$ if $j \in \sD$ and $0$ if $j \not\in D$, and 
$\M_{ij} \in [0,1]$, we have $Z_{ij}(M'_{ij}-\M_{ij}) = |M'_{ij} - \M_{ij}| - \mu v_j(\M'_{ij}-\M_{ij})$, and hence
\begin{align}
\langle M' - \M, Z \rangle &= \sum_{i \in \sC, j} |M'_{ij} - \M_{ij}| - \mu v_j (M'_{ij} - \M_{ij}) \\
 &\geq \|M'-\M|_1 - \mu \sqrt{|\sC|\sum_j v_j^2}\sqrt{\sum_{i \in \sC, j} (M_{ij}'-\M_{ij})^2} \\
 &\geq \|M'-\M\|_1 - \mu \|M' - \M\|_F.
\end{align}
In all, we see that $\|M'-\M\|_1 \leq \epsilon \alpha^2 N^2 + \mu \|M' - \M\|_F \leq 2\epsilon \alpha^2 N^2$.
Since $M' = r(\M)$, this implies the final inequality:
\begin{theorem}
\label{thm:bound}
If $\M$ is the solution to \eqref{eq:optimization-noisy} with $\mu$ and $d$ as defined 
above, then
\[ \sum_{i \in \sC, j} \left|\M_{ij} - \frac{1+A_{ij}}{2}\right| \leq 2\epsilon\alpha^2N^2. \]
In particular, we have
\[ \#\{i \in \sC, j \in [n] \mid \sign(2M_{ij}-1) \neq A_{ij}\} \leq 4\epsilon\alpha^2N^2. \]
\end{theorem}
Let $\round(x) = 1$ if $x \geq \frac{1}{2}$ and $0$ if $x < \frac{1}{2}$. An immediate consequence 
of Theorem~\ref{thm:bound} is that 
there are at least $\frac{1}{2}\alpha N$ rows $i$ for which $\round(A_{ij}) = \bI[j \in \sD]$ 
for all but $8\epsilon \alpha N$ values of $j$. Therefore, if we sample a row randomly, with 
reasonable (i.e., $\frac{\alpha}{2}$) probability that row will tell us all but an $\oo(\epsilon)$ fraction of $\sD$.

\section{Proofs of Lemmas}

\begin{proof}[Proof of Lemma~\ref{lem:concentration}]
We have that 
\[ \sup_{\|M\|_* \leq n^2/\mu} |\langle A-\A, M \rangle| = \frac{n^2}{\mu}\|A-\A\|_{\op}. \]
Therefore, our task is to bound $\|A-\A\|_{\op}$. For this, we make use of \citet{bandeira2014sharp}, 
Theorem 3.1 \& Corollary 3.2, which states that, for any $\epsilon > 0$, we have that
\begin{align}
\bE[\|A-\A\|_{\op}] &\leq (1+\epsilon)\sqrt{\frac{n}{d}}\left\{2\sqrt{n} + 5\sqrt{\frac{\log(n)}{\log(1+\epsilon)}}\right\} \\
 &= \oo\p{\frac{n}{\sqrt{d}}}.
\end{align}
Then, a form of Talagrand's inequality implies that
$\bP[\|A-\A\|_{\op} \geq \bE[\|A-\A\|_{\op}] + t] \leq \exp\p{-c\p{td/n}^2}$. 
Taking $t = (n/d)\sqrt{c^{-1}\log(1/\delta)}$, we have that 
\[ \bP\left[\|A-\A\|_{\op} \geq \oo\p{\frac{n}{\sqrt{d}} + \frac{n}{d}\sqrt{\log(1/\delta)}}\right] \leq \delta, \]
as was to be shown.
\end{proof}

\begin{proof}[Proof of Lemma~\ref{lem:subgradient}]
First, we can take $Z_0 = A - \mu UV^{\top}$, where $U \Lambda V^{\top}$ is the singular 
value decomposition of $M_0$. By assumption, $M_0 = PM_0$, and so $U = PU$ as well. Therefore, 
we do have $(UV^{\top})_{\sC} = \bi v_0^{\top}$ for some vector $v_0$. It remains to show that 
$\|v_0\|_2^2 \leq \frac{1}{|\sC|}$.

To see this, observe that $(UV^{\top})_{\sC}$ is a projection of $UV^{\top}$, and hence 
has operator norm at most $1$. However, because $(UV)^{\top}_{\sC}$ has rank $1$, its 
operator norm and Frobenius norm are equal, and so $\|\bi v_0^{\top}\|_F^2 \leq 1$. 
But $\|\bi v_0^{\top}\|_F^2 = \|\bi\|_2^2\|v_0\|_2^2 = |\sC|\|v_0\|_2^2$, from which 
the result follows.
\end{proof}

\bibliographystyle{plainnat}
\bibliography{references}

\end{document}
