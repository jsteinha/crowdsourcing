\documentclass[11pt]{article}
\usepackage{amsmath,amsfonts,amsthm,amssymb}
\usepackage{hyperref,url}
\usepackage{import,subfiles}
\usepackage{tabularx}
\usepackage{bbm}
\usepackage{algorithm}
\usepackage{algorithmic}
\usepackage{multirow}
\usepackage{fullpage}
\usepackage{natbib}
%\usepackage{hyperref,url}
%\usepackage{amsmath,amssymb}%,amsthm}
%\usepackage{tikz}
%\usepackage{xspace}
%\usepackage{stmaryrd,wasysym}%,clrscode}
%\usepackage{etex,etoolbox}
%\usepackage{ifthen}
%\usetikzlibrary{patterns,positioning}

%%% Added by Jonathan %%%
\def\[#1\]{\begin{align}#1\end{align}}
\def\(#1\){\begin{align*}#1\end{align*}}
\newcommand{\ip}[2]{\left\langle #1, #2 \right\rangle}
\newcommand{\dee}{\mathrm{d}}
\def\argmax{\operatornamewithlimits{arg\,max}}
\def\argmin{\operatornamewithlimits{arg\,min}}
\newcommand{\defined}{\ensuremath{\triangleq}}
\newcommand{\bprf}{\begin{proof}}
\newcommand{\eprf}{\end{proof}}
\newcommand{\blem}{\begin{lemma}}
\newcommand{\elem}{\end{lemma}}
\newcommand{\eps}{\epsilon}
%%% End added by Jonathan %%%

\DeclareMathOperator{\Uniform}{Uniform}
\DeclareMathOperator{\polylog}{polylog}
\newcommand{\oo}{\mathcal{O}}
%\newcommand{\p}[1]{\left(#1\right)}
\DeclareMathOperator{\Geometric}{Geometric}
\newcommand{\op}{\mathrm{op}}
\newcommand{\Beta}{B}
%\DeclareMathOperator{\diag}{diag}
\DeclareMathOperator{\Regret}{Regret}
\newcommand{\algname}[1]{\textsc{\lowercase{#1}}\xspace}
\newcommand{\bmd}{\algname{BMD}}
\newcommand{\primal}{\algname{PBMD}}
\newcommand{\dual}{\algname{DBMD}}
\newcommand{\cgd}{\algname{CGD}}
\newcommand{\md}{\algname{MD}}

\newcommand{\lc}[1]{#1_{\mathrm{loc}}}
\newcommand{\eq}[1]{\stackrel{\mathrm{#1}}{=}}
\DeclareMathOperator{\Var}{Var}
%\DeclareMathOperator{\sign}{sign}
\DeclareMathOperator{\MMD}{MMD}
\newcommand{\MMDr}{\tilde{\MMD}}
%\DeclareMathOperator{\tr}{tr}
%\newcommand{\inner}[2]{\langle #1, #2 \rangle}
%\newcommand{\E}{\mathcal{E}}
%\newcommand{\eqdef}{\stackrel{\mathrm{def}}{=}}
%\newcommand{\bP}{\mathbb{P}}
%\newcommand{\bI}{\mathbb{I}}
%\newcommand{\bE}{\mathbb{E}}
%\newcommand{\sF}{\mathcal{F}}
%\newcommand{\sV}{\mathcal{V}}
%\newcommand{\sH}{\mathcal{H}}
%\newcommand{\sK}{\mathcal{K}}
%\newcommand{\sA}{\mathcal{A}}
%\newcommand{\sC}{\mathcal{C}}
%\newcommand{\sM}{\mathcal{M}}
%\newcommand{\sE}{\mathcal{E}}
%\newcommand{\C}{\mathcal{C}}
%\newcommand{\sB}{\mathcal{B}}
%\newcommand{\bR}{\mathbb{R}}
%\newcommand{\bN}{\mathbb{N}}
%\newcommand{\bZ}{\mathbb{Z}}
%\newcommand{\bF}{\mathbb{F}}
%\newcommand{\sI}{\mathcal{I}}
%\newcommand{\sP}{\mathcal{P}}
%\newcommand{\sX}{\mathcal{X}}
%\newcommand{\sY}{\mathcal{Y}}
%\newcommand{\sZ}{\mathcal{Z}}
%\newcommand{\sS}{\mathcal{S}}
%\newcommand{\sJ}{\mathcal{J}}
%\newcommand{\sR}{\mathcal{R}}
%\newcommand{\sN}{\mathcal{N}}
\newcommand{\meet}{\wedge}
\newcommand{\RE}[2]{\operatorname{RE}\left(#1 \ \| \ #2\right)}
%\newcommand{\KL}[2]{\operatorname{KL}\left(#1 \ \| \ #2\right)}
\newcommand{\KLm}[2]{\operatorname{KL}_m\left(#1 \ \| \ #2\right)}
\newcommand{\score}[2]{\operatorname{score}\left(#1 \ \| \ #2\right)}
\newcommand{\phih}{\hat{\phi}}
\newcommand{\psih}{\hat{\psi}}
\DeclareMathOperator{\supp}{supp}
\DeclareMathOperator{\loc}{loc}
\DeclareMathOperator{\lub}{lub}
\newcommand{\atom}[1]{#1^{\circ}}
\newcommand{\stitch}[2]{\overline{#1}^{#2}}
%\DeclareMathOperator{\argmin}{argmin}
%\DeclareMathOperator{\argmax}{argmax}

%\newtheorem{theorem}{Theorem}[section]
%\newtheorem{lemma}[theorem]{Lemma}
%\newtheorem{proposition}[theorem]{Proposition}
%\newtheorem{corollary}[theorem]{Corollary}
%\newtheorem{assumption}[theorem]{Assumption}
%\theoremstyle{definition}
%\newtheorem{example}[theorem]{Example}
%\newtheorem{definition}[theorem]{Definition}
%\newtheorem{remark}[theorem]{Remark}
%\newtheorem{property}[theorem]{Property}

\def\ci{\perp\!\!\!\perp}

\usepackage{breqn}
\DeclareMathOperator{\Tr}{Tr}
\newcommand{\M}{\tilde{M}}
\newcommand{\A}{\hat{A}}
\DeclareMathOperator{\poly}{poly}
\newcommand{\sD}{\mathcal{D}}
\newcommand{\oo}{\mathcal{O}}
\newcommand{\bi}{\mathbbm{1}}

\begin{document}

\section{Achieving $\poly(\alpha^{-1})$ degree}
We assume we have a matrix $A \in \bR^{n \times n}$ and sets 
$\sC \subset [n], \sD \subset [n]$ satisfying the following:
\begin{itemize}
\item $A_{i,j} = +1$ for all $i \in \sC$, $j \in \sD$
\item $A_{i,j} = -1$ for all $i \in \sC$, $j \not\in \sD$
\end{itemize}
Moreover, assume that $\min(|\sC|, |\sD|) \geq \alpha n$.

Finally, we let $\A$ be a random matrix with i.i.d. entries such that $\bE[\A] = A$, 
and $\bE[(\A_{ij}-A_{ij})^{2p}] \leq \frac{(2p)!}{2^pp!} \p{n/d}^{p}$ for all $p \in \{1,2,\ldots\}$ 
and some fixed $d \in \bR_{>0}$. For instance, this holds if $\Var[\A_{ij}] \leq \frac{n}{d}$ and $|\A_{ij}| \leq \frac{n}{d}$ almost surely, 
which we can obtain by sampling each entry of $A$ with probability $\frac{d}{n}$ and scaling appropriately (which results in $d$ entries in 
each row in expectation).

Let $\M$ be the solution to the following program:
\begin{equation}
\label{eq:optimization-noisy}
\text{maximize } \langle \A, M \rangle - \mu \|M\|_*, \text{ subject to } 0 \leq M_{ij} \leq 1 \, \forall i,j,
\end{equation}
where $\|\cdot\|_*$ denotes the nuclear norm.

We also let $M^*$ be the solution to the ``noiseless'' program
\begin{equation}
\label{eq:optimization-noiseless}
\text{maximize } \langle A, M \rangle - \mu \|M\|_*, \text{ subject to } 0 \leq M_{ij} \leq 1 \, \forall i,j,
\end{equation}

Our goal is to show that $\M$ and $M^*$ are ``close'' in a well-defined sense. We 
will proceed in $3$ steps:
\begin{enumerate}
\item Show that $\M$ is an approximate optimizer of \eqref{eq:optimization-noiseless}.
\item Characterize the optimum $M^*$ of \eqref{eq:optimization-noiseless}.
\item Show that any approximate optimizer of \eqref{eq:optimization-noiseless} is close to $M^*$.
\end{enumerate}

For the first step, we begin with the trivial observation that 
$\max\p{\|\M\|_*, \|M^*\|_*} \leq \frac{n^2}{\mu}$, since otherwise $M = 0$ 
would have a larger objective value for (\ref{eq:optimization-noisy},\ref{eq:optimization-noiseless}). 
We then have the following inequality:
\begin{lemma}
With probability $1-\delta$, we have
\[ |\langle A-\A, M \rangle| \leq \oo\p{\frac{n^3}{\mu}\p{\frac{1}{\sqrt{d}} + \frac{1}{d}\log(1/\delta)}} \]
for all $M$ satisfying $\|M\|_* \leq \frac{n^2}{\mu}$.
\end{lemma}
\begin{proof}
We have that 
\[ \sup_{\|M\|_* \leq n^2/\mu} |\langle A-\A, M \rangle| = \frac{n^2}{\mu}\|A-\A\|_{\op}. \]
Therefore, our task is to bound $\|A-\A\|_{\op}$. For this, we make use of \citet{rmt}, 
Theorem 3.1 \& Corollary 3.2, which states that, for any $\epsilon > 0$, we have that
\begin{align}
\bE[\|A-\A\|_{\op}] &\leq (1+\epsilon)\sqrt{\frac{n}{d}}\left\{2\sqrt{n} + 5\sqrt{\frac{\log(n)}{\log(1+\epsilon)}}\right\} \\
 &= \oo\p{\frac{n}{\sqrt{d}}}.
\end{align}
Then, a form of Talagrand's inequality implies that
$\bP[\|A-\A\|_{\op} \geq \bE[\|A-\A\|_{\op}] + t] \leq \exp\p{-c\p{td/n}^2}$. 
Taking $t = (n/d)\sqrt{c^{-1}\log(1/\delta)}$, we have that 
\[ \bP\left[\|A-\A\|_{\op} \geq \oo\p{\frac{n}{\sqrt{d}} + \frac{n}{d}\log(1/\delta)}\right] \leq \delta, \]
as was to be shown.
\end{proof}
\paragraph{Analyzing $\M$.} Now, we are ready to show that 
$\M$ approximately optimizes \eqref{eq:optimization-noiseless}. 
We will take $\mu = \frac{1}{4}\alpha n$, 
$d = \Omega\p{\max\p{\frac{1}{(\alpha^4\epsilon)^2}, \frac{\log(1/\delta)}{\alpha^4\epsilon}}}$, which yields (for appropriate constants)
\begin{align}
|\langle A-\A, M \rangle| &\leq \frac{1}{2}\epsilon \alpha^2 N^2.
\end{align}
We then have
\begin{align}
\langle A, \M \rangle - \mu \|\M\|_* &\geq \langle \A, \M \rangle - \mu \|\M\|_* - \frac{1}{2}\epsilon \alpha^2 N^2 \\
 &\geq \langle \A, M^* \rangle - \mu \|M^*\|_* - \frac{1}{2} \epsilon \alpha^2 N^2 \\
 &\geq \langle A, M^* \rangle - \mu \|M^*\|_* - \epsilon \alpha^2 N^2.
\end{align}
It follows that $\M$ is within $\epsilon \alpha^2N^2$ of the optimum of \eqref{eq:optimization-noiseless}, 
at least in terms of objective value.


\section{Chracterizing $M^*$}
We prove the following two structural lemmas, which show that 
$M^*$ (approximately) inherits the properties of $A$:
\begin{lemma}
\label{lem:constant}
We have $M^*_{i,j} = M^*_{i',j}$ for all $i,i' \in \sC$.
\end{lemma}

\begin{lemma}
\label{lem:ones}
For all but $\frac{\mu^2}{\alpha N}$ values of $j \in [n]$, we have
$M^*_{i,j} = \frac{1 + A_{i,j}}{2}$.
\end{lemma}

%The advantage of using $\M$ instead of $A$ is that \eqref{eq:optimization} is highly 
%noise-robust. Moreover, given $\M$, we can efficiently identify $\sC$: first sample 
%a vertex at random, and with probability $\frac{\alpha}{4}$ obtain some $v \in \sC_0$. 
%Then, take the set $\sC_1$ of all $j$ such that $M_{v,j} = 1$. Finally, take the set 
%$\tilde{\sC}$ of all $i \in \sC_1$ such that $M_{i,j} = 1$ for all $j \in \sC_1$. 
%We can show that the set $\tilde{\sC}$ is a clique containing $\sC_0$:
%\begin{lemma}
%\label{lem:clique}
%The set $\tilde{\sC}$ is a clique ($M_{i,i'} = 1$ for all $i,i' \in \tilde{\sC}$). 
%Moreover, $\sC_0 \subseteq \tilde{\sC}$.
%\end{lemma}
%\begin{proof}
%First, $\tilde{\sC}$ satisfies $M_{i,j} = 1$ for all $i \in \tilde{\sC}, j \in \sC_1$, 
%and $\tilde{\sC} \subseteq \sC_1$, so in particular $M_{i,j} = 1$ for all $i,j \in \tilde{\sC}$ 
%and hence $\tilde{\sC}$ is a clique.
%
%Second, $\sC_0 \subseteq \sC_1$, and furthermore any $i \in \sC_0$ satisfies 
%$M_{i,j} = M_{v,j} = 1$ for all $j \in \sC_1$, so $\sC_0 \subseteq \tilde{\sC}$.
%\end{proof}

%
%We now prove the two structural lemmas.
\begin{proof}[Proof of Lemma~\ref{lem:constant}]
Consider the matrix $P \in \bR^{n \times n}$ defined by 
$P_{i,i'} = \frac{1}{|\sC|}$ if $i,i' \in \sC$, and 
$P_{i,i'} = \delta_{i,i'}$ otherwise. Then $P$ is clearly a 
projection matrix, and hence $\|PM\|_* \leq \|M\|_*$ for all 
matrices $M$. Furthermore, $\langle A, PM \rangle = \langle A, M \rangle$. 
It follows that replacing $M$ by $PM$ weakly increases the value of 
\eqref{eq:optimization-noiseless}, and hence we must have $M = PM$, from which 
the result follows.
\end{proof}

Before proving Lemma~\ref{lem:ones}, we state and prove the 
following technical lemma which will also be useful later:
\begin{lemma}
\label{lem:subgradient}
Let $f(M) = \langle A, M \rangle - \mu \|M\|_*$, and let 
$M_0$ satisfy $M_0 = PM_0$. Then,
there is a $Z_0 \in \partial f(M_0)$ such that, when restricted 
to the rows lying in $\sC$, we have
\[ (Z_0)_{\sC} = \bi \cdot \left[(2\bi_{\sD}-\bi)^{\top} - \mu v_0^{\top}\right], \]
for some vector $v_0$ of norm at most $\frac{1}{\sqrt{|\sC|}}$.
\end{lemma}
\begin{proof}[Proof of Lemma~\ref{lem:subgradient}]
First, we can take $Z_0 = A - \mu UV^{\top}$, where $U \Lambda V^{\top}$ is the singular 
value decomposition of $M_0$. By assumption, $M_0 = PM_0$, and so $U = PU$ as well. Therefore, 
we do have $(UV^{\top})_{\sC} = \bi v_0^{\top}$ for some vector $v_0$. It remains to show that 
$\|v_0\|_2^2 \leq \frac{1}{|\sC|}$.

To see this, observe that $(UV^{\top})_{\sC}$ is a projection of $UV^{\top}$, and hence 
has operator norm at most $1$. However, because $(UV)^{\top}_{\sC}$ has rank $1$, its 
operator norm and Frobenius norm are equal, and so $\|\bi v_0^{\top}\|_F^2 \leq 1$. 
But $\|\bi v_0^{\top}\|_F^2 = \|\bi\|_2^2\|v_0\|_2^2 = |\sC|\|v_0\|_2^2$, from which 
the result follows.
\end{proof}

\begin{proof}[Proof of Lemma~\ref{lem:ones}]
Let $M^*$ be the optimizer of \eqref{eq:optimization-noiseless}, and 
let $M'$ be $M^*$ with all rows in $\sC$ replaced by $\frac{1+A_{ij}}{2}$.
Note that by concavity of $f$ we have
\[ f(M^*) - f(M') \leq \langle Z', M^* - M' \rangle \]
for any $Z' \in \partial f(M')$. But by Lemma~\ref{lem:subgradient}, we 
have $(Z')_{\sC} = \bi \cdot \left[(2\bi_{\sD}-\bi)^{\top} - \mu v_0^{\top}\right]$, 
where $\|v_0\|_2^2 \leq \frac{1}{\alpha N}$. In particular, $(v_0)_j \geq \frac{1}{\mu}$ 
for at most $\frac{\mu^2}{\alpha N}$ values of $j$. But for all the remaining values of $j$, 
$\sign(Z'_{ij}) = A_{ij}$ for all $i \in \sC$, and hence we must have $M_{ij} = \frac{1 + A_{ij}}{2}$ 
for all $i \in \sC$, as was to be shown.
\end{proof}

\section{Bounding $\M - M^*$}

We have seen that $f(\M) \geq f(M^*) - \epsilon \alpha^2 N^2$. We will now 
use this to show that $\M$ is close to $M^*$. Our argument essentially repeats 
the proof of Lemma~\ref{lem:ones}. Indeed, re-arranging the above 
inequality, we have
\[ \epsilon \alpha^2 N^2 \geq f(M^*) - f(\M) \geq f(M') - f(\M) \geq \langle M' - \M, Z' \rangle, \]
where $Z' \in \partial f(M')$, and $M'$ is defined in terms of $\M$ as above. 
But by Lemma~\ref{lem:subgradient}, we know that $(Z')_{\sC} = \bi \cdot \left[(2\bi_{\sD}-\bi)^{\top} - \mu v_0^{\top}\right]$, 
where $\|v_0\|_2^2 \leq \frac{1}{|\sC|}$. 
\end{document}
