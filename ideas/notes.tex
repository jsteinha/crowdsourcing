\documentclass[11pt]{article}
\usepackage{amsmath,amsfonts,amsthm,amssymb}
\usepackage{hyperref,url}
\usepackage{import,subfiles}
\usepackage{tabularx}
\usepackage{bbm}
\usepackage{algorithm}
\usepackage{algorithmic}
\usepackage{multirow}
\usepackage{fullpage}
\usepackage{natbib}
%\usepackage{hyperref,url}
%\usepackage{amsmath,amssymb}%,amsthm}
%\usepackage{tikz}
%\usepackage{xspace}
%\usepackage{stmaryrd,wasysym}%,clrscode}
%\usepackage{etex,etoolbox}
%\usepackage{ifthen}
%\usetikzlibrary{patterns,positioning}

%%% Added by Jonathan %%%
\def\[#1\]{\begin{align}#1\end{align}}
\def\(#1\){\begin{align*}#1\end{align*}}
\newcommand{\ip}[2]{\left\langle #1, #2 \right\rangle}
\newcommand{\dee}{\mathrm{d}}
\def\argmax{\operatornamewithlimits{arg\,max}}
\def\argmin{\operatornamewithlimits{arg\,min}}
\newcommand{\defined}{\ensuremath{\triangleq}}
\newcommand{\bprf}{\begin{proof}}
\newcommand{\eprf}{\end{proof}}
\newcommand{\blem}{\begin{lemma}}
\newcommand{\elem}{\end{lemma}}
\newcommand{\eps}{\epsilon}
%%% End added by Jonathan %%%

\DeclareMathOperator{\Uniform}{Uniform}
\DeclareMathOperator{\polylog}{polylog}
\newcommand{\oo}{\mathcal{O}}
%\newcommand{\p}[1]{\left(#1\right)}
\DeclareMathOperator{\Geometric}{Geometric}
\newcommand{\op}{\mathrm{op}}
\newcommand{\Beta}{B}
%\DeclareMathOperator{\diag}{diag}
\DeclareMathOperator{\Regret}{Regret}
\newcommand{\algname}[1]{\textsc{\lowercase{#1}}\xspace}
\newcommand{\bmd}{\algname{BMD}}
\newcommand{\primal}{\algname{PBMD}}
\newcommand{\dual}{\algname{DBMD}}
\newcommand{\cgd}{\algname{CGD}}
\newcommand{\md}{\algname{MD}}

\newcommand{\lc}[1]{#1_{\mathrm{loc}}}
\newcommand{\eq}[1]{\stackrel{\mathrm{#1}}{=}}
\DeclareMathOperator{\Var}{Var}
%\DeclareMathOperator{\sign}{sign}
\DeclareMathOperator{\MMD}{MMD}
\newcommand{\MMDr}{\tilde{\MMD}}
%\DeclareMathOperator{\tr}{tr}
%\newcommand{\inner}[2]{\langle #1, #2 \rangle}
%\newcommand{\E}{\mathcal{E}}
%\newcommand{\eqdef}{\stackrel{\mathrm{def}}{=}}
%\newcommand{\bP}{\mathbb{P}}
%\newcommand{\bI}{\mathbb{I}}
%\newcommand{\bE}{\mathbb{E}}
%\newcommand{\sF}{\mathcal{F}}
%\newcommand{\sV}{\mathcal{V}}
%\newcommand{\sH}{\mathcal{H}}
%\newcommand{\sK}{\mathcal{K}}
%\newcommand{\sA}{\mathcal{A}}
%\newcommand{\sC}{\mathcal{C}}
%\newcommand{\sM}{\mathcal{M}}
%\newcommand{\sE}{\mathcal{E}}
%\newcommand{\C}{\mathcal{C}}
%\newcommand{\sB}{\mathcal{B}}
%\newcommand{\bR}{\mathbb{R}}
%\newcommand{\bN}{\mathbb{N}}
%\newcommand{\bZ}{\mathbb{Z}}
%\newcommand{\bF}{\mathbb{F}}
%\newcommand{\sI}{\mathcal{I}}
%\newcommand{\sP}{\mathcal{P}}
%\newcommand{\sX}{\mathcal{X}}
%\newcommand{\sY}{\mathcal{Y}}
%\newcommand{\sZ}{\mathcal{Z}}
%\newcommand{\sS}{\mathcal{S}}
%\newcommand{\sJ}{\mathcal{J}}
%\newcommand{\sR}{\mathcal{R}}
%\newcommand{\sN}{\mathcal{N}}
\newcommand{\meet}{\wedge}
\newcommand{\RE}[2]{\operatorname{RE}\left(#1 \ \| \ #2\right)}
%\newcommand{\KL}[2]{\operatorname{KL}\left(#1 \ \| \ #2\right)}
\newcommand{\KLm}[2]{\operatorname{KL}_m\left(#1 \ \| \ #2\right)}
\newcommand{\score}[2]{\operatorname{score}\left(#1 \ \| \ #2\right)}
\newcommand{\phih}{\hat{\phi}}
\newcommand{\psih}{\hat{\psi}}
\DeclareMathOperator{\supp}{supp}
\DeclareMathOperator{\loc}{loc}
\DeclareMathOperator{\lub}{lub}
\newcommand{\atom}[1]{#1^{\circ}}
\newcommand{\stitch}[2]{\overline{#1}^{#2}}
%\DeclareMathOperator{\argmin}{argmin}
%\DeclareMathOperator{\argmax}{argmax}

%\newtheorem{theorem}{Theorem}[section]
%\newtheorem{lemma}[theorem]{Lemma}
%\newtheorem{proposition}[theorem]{Proposition}
%\newtheorem{corollary}[theorem]{Corollary}
%\newtheorem{assumption}[theorem]{Assumption}
%\theoremstyle{definition}
%\newtheorem{example}[theorem]{Example}
%\newtheorem{definition}[theorem]{Definition}
%\newtheorem{remark}[theorem]{Remark}
%\newtheorem{property}[theorem]{Property}

\def\ci{\perp\!\!\!\perp}

\definecolor{mydarkblue}{rgb}{0,0.08,0.45}
\hypersetup{ %
  pdftitle={},
  pdfauthor={},
  pdfsubject={Proceedings of the International Conference on Machine Learning 2016},
  pdfkeywords={},
  pdfborder=0 0 0,
  pdfpagemode=UseNone,
  colorlinks=true,
  linkcolor=mydarkblue,
  citecolor=mydarkblue,
  filecolor=mydarkblue,
  urlcolor=mydarkblue,
  pdfview=FitH}
\setcitestyle{authoryear,round,citesep={;},aysep={,},yysep={;}}
\usepackage{breqn}
\DeclareMathOperator{\Tr}{Tr}
\newcommand{\M}{\tilde{M}}
\newcommand{\Mm}{M^*}
\newcommand{\A}{\tilde{A}}
\newcommand{\Aa}{A^*}
\DeclareMathOperator{\poly}{poly}
\newcommand{\sD}{\mathcal{D}}
\newcommand{\oo}{\mathcal{O}}
\newcommand{\bi}{\mathbbm{1}}
\newcommand{\todo}[1]{{\color{red} [TODO: {#1}]}}
\DeclareMathOperator{\round}{round}
\DeclareMathOperator{\rank}{rank}

\begin{document}

\section{Assumptions and Main Result}
We assume we have a matrix $A \in \bR^{n \times n}$ and sets 
$\sC \subset [n], \sD \subset [n]$ satisfying the following:
\begin{itemize}
\item $A_{i,j} = +1$ for all $i \in \sC$, $j \in \sD$
\item $A_{i,j} = -1$ for all $i \in \sC$, $j \not\in \sD$
\end{itemize}
Moreover, assume that $\min(|\sC|, |\sD|) \geq \alpha n$.
In addition, we will assume that we have a matrix $\A$ such 
that $\|\A-A\|_{\op} \leq \frac{n}{\sqrt{d}}$ for a parameter 
$d$ to be chosen later. Furthermore, we will assume that for the 
rows indexed by $\sC$, we have $\|\A_{\sC}-A_{\sC}\|_{\op} \leq n\sqrt{\frac{\alpha}{d}}$.

Let $\M$ be the solution to the following program:
\begin{align}
\label{eq:optimization-noisy}
\text{maximize } &\langle \A, M \rangle, \\
\notag \text{ subject to } &M_{ij} \geq 0 \,\,\, \forall i,j, \\
\notag  &\sum_j M_{ij} \leq 1 \,\,\, \forall i, \\
\notag  &\max_j M_{ij} \leq \frac{1}{|\sD|}\sum_j M_{ij} \,\,\, \forall i, \\
\notag  &\|M\|_* \leq \frac{2}{\alpha\epsilon},
\end{align}
where $\|\cdot\|_*$ denotes the nuclear norm.

Let $\Aa$ be the matrix that is equal to $A$ for the rows in $\sC$, 
and equal to $\A$ for the rows outside of $\sC$. Also let $R_{ij} = \frac{1}{|\sD|}\bI[i \in \sC, j \in \sD]$. 
Ideally, we would like to have $M_{\sC} = R_{\sC}$, i.e. $M$ matches $R$ on 
all the rows of $\sC$. In light of this, 
we will let $\Mm$ be the solution to the following ``corrected'' program, which 
we don't have access to (since it involves knowledge of $A$ and $\sC$), but which 
will be useful for analysis purposes:
\begin{align}
\label{eq:optimization-noiseless}
\text{maximize } &\langle \Aa, M \rangle, \\
\notag \text{ subject to } &M_{\sC} = R_{\sC}, \\
\notag  &M_{ij} \geq 0 \,\,\, \forall i,j, \\
\notag  &\sum_j M_{ij} \leq 1 \,\,\, \forall i, \\
\notag  &\max_j M_{ij} \leq \frac{1}{|\sD|}\sum_j M_{ij} \,\,\, \forall i, \\
\notag  &\|M\|_* \leq \frac{2}{\alpha\epsilon}.
\end{align}

Our goal is to show that $\M$ is ``close'' to $M^*$ and hence gives information about $\sC$ and $\sD$. 
We formalize this in the following theorem:
\begin{theorem}
\label{thm:main}
Let $d = \Omega\p{\frac{1}{\alpha^3\epsilon^4}}$. Then, for at least 
half of the rows $i \in \sC$, it is the case that 
$1 - \sum_{j \in \sD} \M_{ij} + \sum_{j \not\in \sD} \M_{ij} \leq \epsilon$.
%$\|\M_i - \pi^*\|_1 \leq \epsilon$, 
%where $\pi^*_j = \frac{1}{|\sD|}$ if $j \in \sD$, and $0$ otherwise.
\end{theorem}
The two key components in proving Theorem~\ref{thm:main} are 
given as propositions below.
We use $\sP$ to denote the constraint set in (\ref{eq:optimization-noisy}).
\begin{proposition}
\label{prop:objective-bound}
Let $d = \Omega\p{\frac{1}{\alpha^3\epsilon^4}}$. For $\M$ and $\Mm$ as defined above, we have
\begin{equation}
\label{eq:objective-bound}
\langle \Aa, \M \rangle \geq \langle \Aa, \Mm \rangle - \epsilon \alpha n.
\end{equation}
\end{proposition}
\begin{proposition}
\label{prop:subgradient}
Let $d = \Omega\p{\frac{1}{\alpha^2\epsilon^2}}$. Then there exists a matrix $Z$ with 
$\rank(Z) = 1$, $\|Z\|_F \leq \epsilon \alpha n$ such that
\begin{align}
\langle \Aa_{\sC} - Z_{\sC}, \Mm_{\sC} - M_{\sC} \rangle &\leq \langle \Aa, \Mm - M \rangle \text{ for all $M \in \sP$}.
\end{align}
\end{proposition}
In the remainder of this section, we will prove Theorem~\ref{thm:main} from Propositions~\ref{prop:objective-bound} 
and \ref{prop:subgradient}. The propositions themselves will be proved in later sections.

\begin{proof}[Proof of Theorem~\ref{thm:main}]
We start by plugging in $\M$ for $M$ in Proposition~\ref{prop:subgradient}. This yields
\begin{align}
\langle \Aa_{\sC} + \mu Z_{\sC}, \Mm_{\sC} - \M_{\sC} \rangle &\leq \langle \Aa, \Mm - \M \rangle \\
 &\leq \epsilon \alpha n
\end{align}
by Proposition~\ref{prop:objective-bound}.
On the other hand, we have 
\begin{align}
|\langle Z_{\sC}, \Mm_{\sC} - \M_{\sC} \rangle| &\leq \|Z_{\sC}\|_F\|\Mm_{\sC} - \M_{\sC}\|_F \\
 &\leq \epsilon\alpha n \sqrt{\sum_{i \in \sC} \|\Mm_i - \M_i\|_2^2} \\
 &\leq \epsilon\alpha n \sqrt{\sum_{i \in \sC} \|\Mm_i - \M_i\|_{\infty}\|\Mm_i - \M_i\|_1} \\
 &= \epsilon\alpha n \sqrt{\frac{\|\Mm_{\sC}-\M_{\sC}\|_1}{\alpha n}} \\
 &\leq \epsilon\alpha n \sqrt{2}.
\end{align}
Putting these together, we obtain
\begin{align}
\langle \Aa_{\sC}, \Mm_{\sC} - \M_{\sC} \rangle &\leq (1+\sqrt{2})\epsilon \alpha n.
\end{align}
Expanding $\langle \Aa_{\sC}, \Mm_{\sC} - \M_{\sC} \rangle$ as 
$\sum_{i \in \sC}\p{\sum_{j \in \sD} (1 - \M_{ij}) + \sum_{j \not\in \sD} \M_{ij}}$, 
we obtain 
\[ \frac{1}{|\sC|}\sum_{i \in \sC}\p{1 + \sum_{j\not\in \sD} \M_{ij} + \sum_{j \in \sD} \M_{ij}} \leq (1+\sqrt{2})\epsilon. \]
The desired result then follows by Markov's inequality (after scaling down $\epsilon$ by a factor of $2\p{1+\sqrt{2}}$).
\end{proof}

\section{Proof of Proposition~\ref{prop:objective-bound}}
By H\"{o}older's inequality, we have 
\begin{align}
|\langle \A-\Aa, M \rangle| &\leq \|\A-\Aa\|_{\op}\|M\|_* \\
 &= \|\A_{\sC}-A_{\sC}\|_{\op}\|M\|_* \\
 &\leq n\sqrt{\frac{\alpha}{d}} \cdot \frac{2}{\epsilon\alpha} \\
 &= \frac{2n}{\epsilon\sqrt{\alpha d}}.
\end{align}
Since $d = \Omega\p{\frac{1}{\alpha^3\epsilon^4}}$, we thus have (for appropriate constants)
\begin{align}
|\langle \A-\Aa, M \rangle| &\leq \frac{1}{2}\epsilon \alpha n.
\end{align}
We then see that
\begin{align}
\langle \Aa, \M \rangle  &\geq \langle \A, \M \rangle - \frac{1}{2}\epsilon \alpha n \\
 &\geq \langle \A, \Mm \rangle - \frac{1}{2} \epsilon \alpha n \text{ (since $\M$ is optimal for $\A$)} \\
 &\geq \langle \Aa, \Mm \rangle - \epsilon \alpha n,
\end{align}
which completes the proof.

%%% TODO --- this is good intuition, should include it somewhere
%%% At a high level, our strategy is the following --- since $\Mm$ is already optimized 
%%% on the rows outside of $\sC$, we should (``morally'', though not in actuality) have 
%%% $\langle \Aa_{\neg \sC}, \Mm_{\neg \sC} \rangle \geq \langle \Aa_{\neg \sC}, \M_{\neg \sC} \rangle$. 
%%% 
%%% Looking back at the rows inside $\sC$, we must therefore have 
%%% $\langle \Aa_{\sC}, \M_{\sC} \rangle \geq \langle \Aa_{\sC}, \Mm_{\sC} \rangle - \epsilon \alpha n$, 
%%% which is exactly the statement \eqref{eq:M-bound} with $(1+\sqrt{2})\epsilon$ replaced by $\epsilon$.
%%% 
%%% The problem with the above argument is that, even though $\Mm$ is optimized on $\neg \sC$, the rows in 
%%% $\sC$ and $\neg \sC$ are coupled by the joint constraint $\|M\|_* \leq \frac{2}{\epsilon\alpha}$. We 
%%% will decouple the rows by taking the subgradient of the Lagrangian at $\Mm$, which yields an upper bound 
%%% for our optimization problem. By doing this we obtain:

\section{Proof of Proposition~\ref{prop:subgradient}}
Let $\sP_0$ be the set $\sP$ where we have removed the nuclear norm constraint. By Lagrangian duality we 
know that there is some $\mu$ such that maximizing $\langle \Aa, M \rangle$ over $\sP \cap \{M_{\sC} = R_{\sC}\}$ 
is equivalent to maximizing $f_\mu(M) \eqdef \langle \Aa, M \rangle + \mu\p{\frac{2}{\epsilon\alpha} - \|M\|_*}$ over 
$\sP_0 \cap \{M_{\sC} = R_{\sC}\}$. 

We start by bounding $\mu$. We claim that $\mu \leq \epsilon \alpha n$. 
To show this, first note that for any $M \in \sP_0$, we have 
\begin{align}
\langle \Aa, M \rangle &\leq \langle A, M \rangle + \langle \Aa - A, M \rangle \\
 &\leq n + \|\Aa - A\|_{\op}\|M\|_* \\
 &\leq n + \frac{n}{\sqrt{d}}\|M\|_* \\
 &\leq n + \frac{\epsilon\alpha n}{2}\|M\|_*,
\end{align}
where in the last step we used the assumption $d = \Omega\p{\frac{1}{\alpha^2\epsilon^2}}$.

Now, suppose that we take $\mu_0 = \epsilon \alpha n$ and optimize $\langle \Aa, M \rangle - \mu_0\|M\|_*$ over 
$\sP_0 \cap \{M_{\sC} = R_{\sC}\}$. Since $\langle \Aa, M \rangle - \mu_0\|M\|_* \leq n - \frac{\epsilon \alpha n}{2}\|M\|_*$, 
any $M$ with $\|M\|_* > \frac{2}{\epsilon\alpha}$ cannot possibly be optimal, since the solution $M = 0$ would 
be better. Hence, $\mu_0$ is a large enough Lagrange multiplier to ensure that $M \in \sP$, and so 
$\mu \leq \mu_0 = \epsilon \alpha n$, as claimed.

We next characterize the subgradient of $f_{\mu}$ at $M = \Mm$.
Define the projection matrix $P$ as
\[ P_{i,i'} = \left\{ \begin{array}{ccl} \frac{1}{|\sC|} & : & i, i' \in \sC \\ \delta_{i,i'} & : \text{else} \end{array} \right.. \]
Thus $PM = M$ if and only if all rows in $\sC$ are equal to each other.
In particular, $PM = M$ whenever $M_{\sC} = R_{\sC}$. Now, since $\Mm$ is the maximum 
of $f_{\mu}(M)$ over all $M \in \sP_0 \cap \{M_{\sC} = R_{\sC}\}$, there must be some 
$B \in \partial f(\Mm)$ such that $\langle B, M - \Mm \rangle \leq 0$ for all $M \in \sP_0 \cap \{M_{\sC} = R_{\sC}\}$.
\begin{lemma}
\label{lem:subgradient}
Suppose that $B \in \partial f(\Mm)$ satisfies $\langle B, M - \Mm \rangle \leq 0$ for all $M \in \sP_0 \cap \{M_{\sC} = R_{\sC}\}$. 
Then, $PB$ satisfies the same property, and lies in $\partial f(\Mm)$ as well.
\end{lemma}
We can further note (by differentiating $f_{\mu}$) that 
$B = \Aa - \mu Z_0$, where $\|Z_0\|_{\op} \leq 1$. Hence 
$PB = P\Aa - \mu PZ_0 = \Aa - \mu PZ_0$. Let $r(M)$ denote the 
matrix where $M_{\sC}$ is replaced with $R_{\sC}$ (so $r(M) \in \sP_0 \cap \{R_{\sC} = M_{\sC}\}$ 
whenever $M \in \sP_0$). The rest of the proof is basically algebra:
\begin{align}
\langle \Aa, M - \Mm \rangle &\stackrel{(i)}{\leq} f_{\mu}(M) - f_{\mu}(\Mm) \\
 &\stackrel{(ii)}{\leq} \langle \Aa - \mu PZ_0, M - \Mm \rangle \\
 &= \langle \Aa - \mu PZ_0, M - r(M) \rangle + \langle \Aa - \mu PZ_0, r(M) - \Mm \rangle \\
 &\stackrel{(iii)}{\leq} \langle \Aa - \mu PZ_0, M - r(M) \rangle \\
 &= \langle \Aa_{\sC} - \mu (PZ_0)_{\sC}, M_{\sC} - r(M)_{\sC} \rangle \\
 &= \langle \Aa_{\sC} - \mu (PZ_0)_{\sC}, M_{\sC} - \Mm_{\sC} \rangle,
\end{align}
where (i) is by complementary slackness (either $\mu = 0$ or $\|\Mm\|_* = \frac{2}{\alpha\epsilon}$); 
(ii) is concavity of $f_{\mu}$, and the fact that $\Aa - \mu PZ_0$ is a subgradient; and 
(iii) is the property from Lemma~\ref{lem:subgradient} ($\langle \Aa - \mu PZ_0, r(M) - \Mm \rangle \leq 0$ since 
$r(M) \in \sP_0$).

To finish, we will take $Z = \mu (PZ_0)_{\sC}$. We note that
$\|Z\|_{\op} = \|\mu (PZ)_{\sC}\|_{\op} \leq \mu \|PZ\|_{\op} \leq \mu \|Z\|_{\op} \leq \mu$.
Moreover, $Z$ has rank $1$ and so $\|Z\|_F = \|Z\|_{\op} \leq \mu \leq \epsilon\alpha n$, as was to be shown.

\subsection{Proof of Lemma~\ref{lem:subgradient}}
First, since $PM = M$ for all $M \in \sP_0 \cap \{M_{\sC} = R_{\sC}\}$, and $PM^* = M^*$, 
it is clear that $\langle PB, M-M^* \rangle = \langle B, P(M-M^*) \rangle = \langle B, M-M^* \rangle \leq 0$, 
so we need only show that $PB$ is still a subgradient of $f_{\mu}$. Indeed, we have (for any $M$)
\begin{align}
\langle PB, M-M^* \rangle &= \langle B, PM - M^* \rangle \\
 &\stackrel{(i)}{\geq} f_{\mu}(PM) - f_{\mu}(M^*) \\
 &= \langle A^*, PM \rangle - \mu \|PM\|_* - f_{\mu}(M^*) \\
 &= \langle A^*, M \rangle - \mu \|PM\|_* - f_{\mu}(M^*) \\
 &\stackrel{(ii)}{\geq} \langle A^*, M \rangle - \mu \|M\|_* - f_{\mu}(M^*) \\
 &= f_{\mu}(M) - f_{\mu}(M^*),
\end{align}
where (i) is because $B \in \partial f_{\mu}(M^*)$, and (ii) is because projecting 
decreases the nuclear norm. Since the inequality $\langle PB, M-M^* \rangle \geq f_{\mu}(M) - f_{\mu}(M^*)$ 
is the defining property of $\partial f_{\mu}(M^*)$, the proof is complete.

%\bibliographystyle{plainnat}
%\bibliography{references}

\end{document}
